\section{Background}

% TODO: text här? Upplägg?

\subsection{Stack traces}

When a computer program crashes, the runtime of some programming languages
gives some context to where in the code the program crashed.
Typically, a \emph{stack trace} is printed. A stack trace is the listing
of the functions that have called each other and have not exited yet, so
they have all been part of the crash. The first function in the stack
trace is always the program entry point, the last function is where the
crash actually occurred.

% TODO: Add example or reference

\subsection{Haskell}

Haskell is a lazy, functional, general-purpose programming language.
\cite{haskell_report2010}
It first appeared in 1990 \cite{HistoryOfHaskell2007}  and have since
released the major standards Haskell 98 and Haskell 2010
\cite{haskell_report2010}. Figure \ref{fig:simple_program} shows a simple program in Haskell.

\begin{figure}
  \begin{minted}{haskell}
main = print (fibonacci 10)

fibonacci :: Int -> Int
fibonacci 1 = 0
fibonacci 2 = 1
fibonacci x = fibonacci (x - 1) + fibonacci (x - 2)
  \end{minted}
  \caption{A simple Haskell program}
  \label{fig:simple_program}
\end{figure}


Two functions are defined in this program, \texttt{main} and
\texttt{fibonacci}.  The explicit type signature for the function
\texttt{fibonacci} means that it takes an int and returns an int. If a type
signature is omitted, like for \texttt{main}, Haskell will infer it automatically.

\subsubsection{Error handling in Haskell}

In order for stack traces to be relevant for a programming language, programs
must have the notion of \emph{crashing}. Intuitively, crashing causes sudden
stops in execution, either by the operating system or by the language's own
exception handling. Program crashes can be disastrous, since they will also
terminate processes that are supposed to be long-running. Hence there are
language constructs to eliminate some causes of program crashes.
For instance,
in well-typed Haskell programs, there are no segfaults \cite{FindingTheNeedle2009}.
% TODO: (behövs? källa?)   or any other time where the operating system
% directly intervenes. (really? source?)

In Haskell, there's a notion of a function being \emph{total}. Meaning that a
function will terminate and not return any error. % TODO: (source to definition?)
Therefor, such a function can not crash. Unfortunately, as of the famous
halting problem it's not possible to decide if a function will terminate or not, so it's not possible to say if a function is total.
% TODO (source: some really old paper by Turing?)
The good news, though, is that whenever we
explicitly \emph{choose} to crash, we can systematically avoid it. This is not
only true in the language Haskell, Java implements this through the
\texttt{throws} keyword. %TODO citera?
Figure \ref{fig:total_java} shows a Java integer division function that's
total.

\begin{figure}
  \begin{subfigure}[t]{1\textwidth}
      \begin{minted}{java}
int integerDivision (int nom, int den) throws ArithmeticException {
  if (den == 0) {
    throws ArithmeticException("Division by zero");
  }
  else {
    return nom / den;
  }
}
       \end{minted}
    \caption{A total function in Java}
    \label{fig:total_java}
  \end{subfigure}
        \vskip2em
        \begin{subfigure}[t]{1\textwidth}
          \begin{minted}{haskell}
integerDivision :: Int -> Int -> Maybe Int
integerDivision n 0 = Nothing
integerdivision n d = Just (n `div` d)
          \end{minted}
          \caption{A total function in Haskell}
          \label{fig:total_haskell}
        \end{subfigure}
        \caption{Two total functions
        }\label{fig:total_functions}
\end{figure}


In Haskell, it's instead convention to simply not return a value when there's
no valid value to return. To do this in Haskell, the \emph{Maybe} wrapper is
used to allow functions to gracefully fail, as illustrated in figure
\ref{fig:total_haskell}.
% TODO, eventuellt jämföra med Pythons None

The two functions in figure \ref{fig:total_functions} will not crash when dividing
by zero, rather, they gracefully return a value of either the division or a
value representing failure. But there's a drawback, both these functions are
cumbersome to use. In Java the programmer needs to explicitly catch the
Exception % TODO (citera?)
combining the \texttt{try} and \texttt{catch} constructs.
\cite{oracle_java_doc_catch} In Haskell, an additional layer of pattern
matching is required. Due to this inconvenience, both languages allow for
carrying out integer division without forcing the caller to do any error
handling. Figure \ref{fig:partial_functions} shows two partial functions.
% TODO fotnot om att det i java kallas RuntimeException när det intebehöver
% fångas?
% Alternativt kan man ju faktiskt ge länken till kontreversin
% http://docs.oracle.com/javase/tutorial/essential/exceptions/runtime.html
% om unchecked exceptions. För att visa att det jag pratar om är en riktig
% grej.

\begin{figure}
  \begin{subfigure}[t]{1\textwidth}
      \begin{minted}{java}
int integerDivisionUnsafe (int nom, int den) {
  return nom / den;
}
      \end{minted}
    \caption{A partial function in Java}
    \label{fig:partial_java}
  \end{subfigure}
        \vskip2em
        \begin{subfigure}[t]{1\textwidth}
          \begin{minted}{haskell}
integerDivisionUnsafe :: Int -> Int -> Int
integerDivisionUnsafe n 0 = error "Division by zero"
integerDivisionUnsafe n d = n `div` d
          \end{minted}
          \caption{A partial function in Haskell}
          \label{fig:partial_haskell}
        \end{subfigure}
        \caption{Two partial functions
        }\label{fig:partial_functions}
\end{figure}

For the first time we now see the \texttt{error} function in Haskell (figure \ref{fig:partial_haskell}).  It's a
special in-built function that terminates execution and outputs the provided
message. While it's not entirely accurate, we could think of \texttt{error}
being the only gateway to crashing a Haskell program. That means that all the
typical dangerous operations like integer division by zero or indexing outside
an array would just invoke the \texttt{error} function. We define ``crashing''
to be whenenver \texttt{error} is called.

% (footnote: It's
% rather the evaluation of an expression tree having \texttt{error} as it's top
% expression). % TODO: Behövs inte kanske.

% TODO: (ska man ha en formell Definition 1.1 blah blah här?)

\subsection{Glasgow Haskell Compiler}

The Glasgow Haskell Compiler (GHC) is a Haskell2010 compatible compiler.
\cite{ghc_website} With it you can compile Haskell source code to an executable
binary. Here's an invocation of the compiler on the program sample from figure \ref{fig:simple}.

\begin{minted}{bash}
  $ ghc --make Fibonacci.hs
  ...
  $ ./a.out
  34
\end{minted}

% GHC is the by far most used Haskell compiler (source). It has always (?)
% been in active (?) development since its first release in 199x (source) (Onödigt? Ta
% bort allt till och med hit?). Since then, many notable extensions have been
% added.
GHC as of today support many features in addition to the Haskell2010
standard, like parallelism, many optimizations, an llvm backend, profiling
support and more \cite{ghc_website}. This
work is adding stack traces to that list.

\subsubsection{The execution stack}

Most programmers have a decent idea of how programming languages implement
functions. They use the stack pointer register. Whenever a function is called, its
arguments are pushed on the stack by the caller and the caller jumps to the
functions code. When the function finally exits, it returns to where it was
called from. This was a short reminder of how the \emph{regular stack} works.
Most programming languages use this to implement the concept of functions.

Due to the nature of Haskell. It's not clear if the stack that worked so
elegantly for languages like C can be used to implement Haskell. How does it
work with partial applications? How does it work for lazyness? Instead one
might look at creating a completly new execution machine not based on a stack.
Peyton Jones created the Spineless Tagless G-Machine \cite{stg_1992}, an
execution machine which Haskell can be compiled down to. This machine \emph{is}
actually inspired by in the typical stack model. It's stack is called the
\emph{execution stack} and GHC implements this machine \cite{evalapplyjfp06}.
It's similar to stacks in typical programming languages. One property of this
stack is that a function written in the original source language (the Haskell
code) will have its own stack frame on the stack. Again, this is not a
necessity for a Haskell implementation. But it is the whole basis of my thesis.
If stack traces are going to be reconstructed by walking the stack, a stack
must exist! The take away here is that when we're talking about walking the
stack in this paper, we mean the execution stack as described in the STG, not
the traditional stack. Therefore my work only applies to GHC and not Haskell in
general.

% TODO: (bra bild på stacken)

% TODO: Att det är en länkad lista kanske ska komma senare I nästa del? Eller
% ska den komma här? Troligen här va? Typ som i Tereis paper

% \subsection{DWARF}

% (vad ska jag skriva här? Ännu oklart hur mycket DWARFmekande jag själv kommer
% göra under arbetets gång ...)

% Skriva om varför debugging är svårt? Inspiration från DWARF-summarien
% Sedan vidare skriva om peters work, hans "causality" och sånt samt att allt
% egentligen är en best effort.

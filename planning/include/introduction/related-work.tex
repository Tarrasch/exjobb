\section{Related work}

For GHC, the default settings do not print stack traces on errors. While there
are many successful stack trace implementations in Haskell, they do have a
performance penalty since they don't use the actual \emph{execution stack}.  This
performance penalty leaves them out of being a default option of ghc (källa?
påhittat?).
On the other hand the execution stack is harder to interpret, in this section
we will look at existing work on the 
follows a listing of previous work about stack traces for Haskell who all
essentially work with maintaining a separate structure representing the stack
that's passed along during execution.

\subsection{Haskell data visualization}

(Är detta område intressant och relevant? Jag tänker saker såsom
\href{http://felsin9.de/nnis/ghc-vis/}{ghc-vis}.)

\subsection{Overhead-full stack traces}

\begin{figure}
        \begin{subfigure}[t]{0.5\textwidth}
            \begin{minted}{haskell}
main = print (f 100)

f :: Int -> Int
f x = g (5*x)

g :: Int -> Int
g 7 = error "Bang"
g x = 100 * x
            \end{minted}
            \caption{Original program}
            % \label{fig:gull}
        \end{subfigure}
        ~ %add desired spacing between images, e. g. ~, \quad, \qquad etc.
          %(or a blank line to force the subfigure onto a new line)
        \begin{subfigure}[t]{0.5\textwidth}
          \begin{minted}{haskell}
main = print (f stack' 100)
  where
    stack' = "main \n"

f :: String -> Int -> Int
f stack x = g stack' (5*x)
  where
    stack' = "f (case 1)\n" ++ stack

g :: String -> Int -> Int
g stack 7 = error ("Bang" ++ stack')
  where
    stack' = "g (case 1)\n" ++ stack
g stack x = 100 * x
  where
    stack' = "g (case 2)\n" ++ stack
          \end{minted}
          \caption{Transformed program}
          % \label{fig:tiger}
        \end{subfigure}
        \caption{An example of how a Haskell program can be transformed to one
          that will print stack traces on errors. The syntax `\texttt{str1 ++
            str2}' is string concatenation.
        }\label{fig:transformation}
\end{figure}

Stack traces can be achieved by doing some methodological source level
transformations. Figure \ref{fig:transformation} shows a program transformed
into one producing stack traces on calls to \texttt{error}. This transformation is essentially:

\begin{itemize}
\itemsep1pt\parskip0pt\parsep0pt
\item
  Changing all top level functions to take one additional string
  argument. Except for the program entry-point \texttt{main}.
\item
  Transform all equations to define the new call stack \texttt{stack'} and
  pass it as the first argument to all calls of top level functions.
\item
  Transform all calls to \texttt{error} to also print out the call stack.
\end{itemize}

This transformation is similar to
`http://ghc.haskell.org/trac/ghc/wiki/ExplicitCallStack\#Transformationoption1'
(9th oct 2013). While the idea is simple there are many complications and
drawbacks as noted by other implementors.

Allwood et al implemented this in GHC (source: Finding the needle) . Among it's complications are the
handling of higher order functions, linking with code that doesn't have stack
traces and most important of all, an efficient non-naive implementation of the
passed along stack. Functional programming in particular relies on efficient
tail call optimizations(source), which requires the passed around call stack to
efficiently handle this.

The other dimensionality of complications is how there is a more general
picture other than stack traces. There are other debugging aspects a compiler
would like to support, and the more features the bigger blowup in the
complexity of the compiler itself.  Clements disseratation shows a framework
called \emph{continuation marks} which allows for a generic way to create
addons to the language (source: \href{http://www.brinckerhoff.org/clements/papers/index.html}{Clements homepage}).
In his disseratation, he uses continuation marks as a
common ground of implementation for stack traces, code stepping in debugging
and aspect oriented programming.  Tough no haskell implementation of
continuation marks have been implemented.

The currently most stable implementation of Overhead-full stack traces is 
(Simon marlow) % TODO(arash)

\subsection{Other work on Stack Traces for Haskell}

At the start of this thesis, Peter Wortmann had already ... % TODO

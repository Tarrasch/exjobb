\chapter{Reifying the Stack}

In the previous chapter we looked at how the stack works, in its
last section we saw the process of reifying the stack as it was
implemented in the Peter Wortmann's demonstration in August 2013
\cite{stack_traces_ticket}. This chapter will be dedicated to examine
the when and how of stack reification in detail. We look at the problems
of the approach from the prototype we find room both for improvement and
discussion.

In section 5.1 TODO we see that the stack traces can become more
readable by using the extra information in the payloads of each stack
frame. Section 5.2 deals with the serious issue of wasting resources by
reifying the stack without ever using it.

\section{Frames of interest}

Update frames yay

\section{Lazy reification}

TODO:
But this looks efficient! We only add an overhead by collecting the
stack into a seperate array and then we just print it. Non of these is
trivially avoidable! However, Haskell programmers might install handlers
that choose to not print the value.

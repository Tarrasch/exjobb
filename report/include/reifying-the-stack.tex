\chapter{Reifying the Stack} \label{chp:reifying_the_stack}

In section~\ref{sec:recent_work} we saw Peter Wortmann's prototype for
execution stack based stack traces. It was demonstrated in August
2013 \cite{stack_traces_ticket}. This is a rough sketch of how it is
implemented:

\begin{enumerate}
  \item
    The program starts by installing a catch-all handler. This handler will
    print the stack trace on a crash.
  \item
    Program runs and crashes.
  \item
    The run time system handles the crash by walking through the whole
    execution stack and saving it in a separate array. It then invokes the
    installed handler and passes the array.
\end{enumerate}

To store the essential contents of the stack in a new value,
possibly for later trace-printing, is called \emph{reifying} the
stack. The reification from Peter Wortmann's demonstration is
quite simple, figure~\ref{fig:reification} illustrates
this method of reifying the stack.

\begin{figure}
\begin{mdframed}
  \includegraphics[width=5in]{build/fig/reification}
  \caption{An illustration of stack reification, the rightmost box is the array
    allocated to store the result of the reification, the diamond arrows show
    where the content is copied from.}
  \label{fig:reification}
\end{mdframed}
\end{figure}

This chapter will go into
stack reification in detail. By critically looking
at the prototype, we find room both for improvement and discussion.
In section~\ref{sec:frames_of_interest} we see that the stack traces can become more
readable by using the extra information in the payloads of each stack
frame. Section~\ref{sec:lazy_reification} deals with the issue of wasting
resources when reifying the stack without ever using it.

\section{Frames of interest} \label{sec:frames_of_interest}

Figure~\ref{fig:traces} showed that the stack trace from Peter Wortmann's
demonstration is very far from the ideal stack trace.  Worse, the only
information we have to work with is the execution stack, remember, we maintain no
explicit call stack for performance reasons. Still, the stack can become
clearer by using the payload of the stack frames. For the readers convenience,
we reproduce the stack trace here again as figure~\ref{fig:traces_2}.

\begin{figure}
\begin{mdframed}
  \begin{minted}[gobble=4]{text}
     0: stg_bh_upd_frame_ret
     1: stg_bh_upd_frame_ret
     2: stg_bh_upd_frame_ret
     3: showSignedInt
     4: stg_upd_frame_ret
     5: writeBlocks
     6: stg_ap_v_ret
     7: bindIO
     8: bindIO
     9: bindIO
    10: bindIO
    11: stg_catch_frame_ret
  \end{minted}
        \caption{Stack trace (same as in figure~\ref{fig:trace_goal1}).}
        \label{fig:traces_2}
\end{mdframed}
\end{figure}

In section~\ref{sec:update_frames} and~\ref{sec:the_other_frames}
we will revisit the stack frames we looked at in
section~\ref{sec:members_of_stack}. This time we will look at how
their frame-specific payload can be utilized to make the stack
traces more useful. In addition to utilizing the existing frames, we
take a stab at creating our own artificial stack frame in
section~\ref{sec:artificial_frames} that could improve stack traces.

\subsection{Update frames} \label{sec:update_frames}

Frame 0,1,2 and 4 from figure~\ref{fig:traces_2} are all update
frames. Recall that the code for a thunk will push one update
frame. Thunks are common in Haskell and therefore update frames are common
as well. If we were able to print something better
than the cryptic \texttt{stg\_bh\_upd\_frame} and \texttt{stg\_upd\_frame},
many frames in the stack trace would improve.

The actual code of the thunk that pushed the update frame
could be anything. To only see the info pointer of the update frames is hardly helpful.
But we know that the field of the update frame is a reference to the
heap object the update frame is going to update, the \emph{updatee}. The updatee's
info table contains the code that pushed the update frame.
This is really good, because we can copy that info pointer to our array instead,
as illustrated in figure~\ref{fig:reify_update_frame}.
Unfortunately this trick only works in some cases and it will depend on the kind of
update frame we have on the stack. There are three different kinds of
update frames \cite{github_updates_cmm}.

\begin{figure}
\begin{mdframed}
  \includegraphics[width=5in]{build/fig/reify_update_frame}
  \caption{An improved way of reifying update frames. The old method is showed
    in the dashed line.}
  \label{fig:reify_update_frame}
\end{mdframed}
\end{figure}

\begin{itemize}
  \item
    \texttt{stg\_bh\_upd\_frame} is the update frame used for global
      thunks. Global thunks are better known as CAFs.
  \item
    \texttt{stg\_upd\_frame} is the update frame used for local
  thunks.
  \item
     \texttt{stg\_marked\_upd\_frame} is created
  by overwriting one of the other two kinds of frames, it
  happens during garbage collection and this phase is named
  \emph{blackholing}. Blackholing converts any update frame to a
  marked update frame \cite{github_overwrite_update_frame} and
  overwrites the updatee's info pointer to point to a "black hole"
  \cite{github_overwrite_blackhole}.
\end{itemize}

Unfortunately, only the updatee of local thunks point
at interesting code and not a black hole. The marked update frame
always gets its updatee's info pointer overwritten to a black
hole \cite{github_overwrite_blackhole} and the update frame
for global thunks' updatee point at a black hole to begin with
\cite{github_set_hdr_caf_blackhole}.\footnote{The names
  \texttt{stg\_bh\_upd\_frame} and \texttt{stg\_marked\_upd\_frame} are very
  confusing. In both cases they point at black holes. A
  \texttt{stg\_marked\_upd\_frame} points at a \texttt{BLACKHOLE} and a
  \texttt{stg\_bh\_upd\_frame} points at a \texttt{CAF\_BLACKHOLE}. The names
  are kept true to their name in the GHC source code to ease verifiability.}


Blackholing have become a quite complicated part of the run-time system,
having multiple purposes \cite{ghc_new_implementation_black_holes}
and is implemented with low-level tricks like pointer tagging. But the
traditional black hole as described in \cite{jones1992tail} is having
clear purposes, blackholing fixes a class of space leaks and it detects
some cases of nontermination. In concurrent Haskell, blackholing have
synchronization purposes that increases sharing (avoids re-computation). Blackholing
is illustrated in figure~\ref{fig:thunks_and_blackholes}.

\begin{figure}
\begin{mdframed}
  \begin{subfigure}[t]{0.5\textwidth}
    \includegraphics[width=2.6in]{build/fig/thunk}
    \caption{A thunk.}
  \end{subfigure}
        ~ %add desired spacing between images, e. g. ~, \quad, \qquad etc.
          %(or a blank line to force the subfigure onto a new line)
  \begin{subfigure}[t]{0.5\textwidth}
    \includegraphics[width=2.8in]{build/fig/blackhole}
    \caption{A blackhole.}
  \end{subfigure}
  \caption{The same thunk, before and after blackholing.}\label{fig:thunks_and_blackholes}
\end{mdframed}
\end{figure}

\subsubsection{Retaining code reference on blackholing}

The bad effect of blackholing from a stack trace perspective is
that thunks lose the reference to the code that pushed the update frame (recall figure~\ref{fig:reify_update_frame} and figure~\ref{fig:thunks_and_blackholes}). So instead
of being able to print something useful, like  \texttt{... 4: print ...},
we have to print \texttt{... 4: stg\_upd\_frame\_ret ...}.

Unfortunately, blackholing is not optional
\cite{github_blackholing_not_optional}. It is however possible to
retain the reference by just adding a field for all thunks and
copying the reference there. The extended field is shown in figure~\ref{fig:different_layout}.

\begin{figure}
\begin{mdframed}
  \begin{subfigure}[t]{1\textwidth}
    \includegraphics[width=5.5in]{build/fig/old_thunk}
    \caption{The current layout.}
  \end{subfigure}
  \vskip2em
  \begin{subfigure}[t]{1\textwidth}
    \includegraphics[width=5.5in]{build/fig/new_thunk}
    \caption{A layout that has a copy of the original info pointer, since the
      first info pointer will be overwritten during blackholing.}
  \end{subfigure}
  \caption{Two different layout for thunks.}\label{fig:different_layout}
\end{mdframed}
\end{figure}

The extra field has a runtime cost of course. The number of thunks
ever created during the lifetime of a Haskell program has no upper bound. However,
there is only a constant number of global thunks, which means that the
performance cost would be insignificant if we only applied the idea from
figure~\ref{fig:different_layout} to global thunks. When
only changing the global thunks, we are able to identify
the \texttt{stg\_bh\_upd\_frame} frames but not the
\texttt{stg\_marked\_upd\_frame} frames. For example, the
segment

\begin{minted}[gobble=4]{text}
     0: stg_bh_upd_frame_ret
     1: stg_bh_upd_frame_ret
     2: stg_bh_upd_frame_ret
     ...
\end{minted}

could instead be

\begin{minted}[gobble=4]{text}
     0: divZeroError
     1: crashSelf
     2: c
     ...
\end{minted}

by utilizing the new payload. The second stack trace is of course
\emph{much} more useful than the first stack trace. Unfortunately,
to only add the field for global thunks will only be useful for a
short time of the life of the thunk, since blackholing is happening
intermittently. If instead all thunks had the extra field, the blackhole
thunks would also retain the useful reference. Luckily though, if a
global thunk is containing code that crashes, it is probable that its
update frame will be left intact if it crashes early. In
fact, we should consider ourselves extra lucky that top-level values
like \texttt{divZeroError} memoize. If there were no memoization,
there would be no thunk or no update frames, instead there would be
regular tail calls and the first three frames would not even be on the stack.

\subsection{The other frames} \label{sec:the_other_frames}

Some of the stack frames don't need any analysis because their info
pointer is what we want to print, other frames seem hopeless and are not
investigated in depth. In this subsection we will give a brief mention
of the frames that we have not analyzed in depth.

For \emph{case continuations}, the info pointer is pointing to code
generated from Haskell, so we consider case continuations as a
trivial frame. For \emph{Call continuations}, we've already
jumped to the interesting code and just left a trace of leftover arguments.
There is nothing to do for call continuations since the leftover arguments
say nothing about the over-applied function itself.
\emph{Catch frames} will hold a reference to the handler, but it is
not what is currently evaluating. The good news is that just seeing
any catch frame is helpful, with some context the programmer might be
able to determine which \texttt{catch} in the source code pushed the
catch frame.

\subsection{Artificial frames} \label{sec:artificial_frames}

So far we have looked at how to utilize the payload of \emph{already existing} stack
frames. In this chapter we look at how we can add new \emph{artificial}
stack frames.
Artificial frames are stack frames whose only purpose is to improve
the stack trace. The intended use case is for programmers that have
got a reproducible bug and a stack trace without enough information.
This should be a common situation, because stack traces derived from the
execution stack often lack frames due to tail calls. A versatile
programmer might switch to one of the many other stack trace
implementations from section~\ref{sec:overhead_full}.
But in practice that is usually an overwhelming task
due to technical difficulties like compiler flags and missing libraries. Instead we present
the programmers with an inbuilt primitive to force a stack frame.
When a stack is later reified, these artificial frames would work as
checkpoints and be printed.

There are two fundamentally different approaches here. Either you create
an actual function like \texttt{pushStackFrame :: a -> a} that takes
one argument, pushes its arguments info pointer on the execution stack
(encapsulated in a special stack frame) and then evaluates the argument.
The other approach is to create a ``macro''-like function that will
expand \texttt{(forceCaseContinuation x)} to \texttt{(case x of \_ ->
x)} in such a way that the case expression can not be optimized away.
This approach will create a unique case continuation for every use site
of \texttt{forceCaseContinuation}. The fundamental difference is that
the first approach will save the \emph{jump target} in the artificial
frame's field. Meanwhile the second approach pushes a case continuation
that will uniquely identify the \emph{jump source}. It is not obvious
which method will be most useful and do what the programmer expects. One
nice property of the jump source implementation is that it will include
function $f$ in the stack trace if \texttt{forceCaseContinuation} is
put inside function $f$ and its argument is currently evaluating.
Further, the jump target approach has a pitfall: When a nontrivial
expression is passed to it, the nontrivial expression will be put into
a thunk that will have a source location associated to it from the jump
source. So the jump target will often be the same as
the jump source. The programmers using \texttt{pushStackFrame} should
therefore understand when GHC decides to create thunks. On the other
hand, \texttt{pushStackFrame} can be the right tool for the job and will
push an artificial stack frame containing an interesting jump target.

\section{Efficient reification} \label{sec:lazy_reification}

When executing the program from figure~\ref{fig:sample_program}, the reification need to happen at the crash site and
not in the handler. When control has been passed to the handler the
execution stack has been unrolled already, so the stack is inaccessible
and maybe even overwritten. At the same time, there is a cost associated
with reifying the stack and the cost is growing linearly with the size
of the execution stack. So when we \emph{know} that we're going to print
the stack, it is acceptable to have the additional linear cost of
reifying the stack, but we shouldn't tolerate the cost when we are not
using the reified stack value. One consequence of always reifying the
stack is that functions that use \texttt{throw} for control flow will
become slower depending on how big the stack is when they're called.
To have the time complexity of a total Haskell function depending on
the state of the underlying runtime system is not acceptable for a mature
compiler like GHC. Note that control flow can be an anticipated use case for
exceptions \cite{peyton1999semantics} (however, a more recent paper
does not anticipate it \cite{marlow2006extensible}). \texttt{lazyReadBuffered} uses exception handling
for control flow \cite{github_lazyReadBuffered_control_flow}.
In this section we will look for alternative solutions
to unconditional and strict stack reification.

Before looking at theoretical solutions, we will test our hypothesis
and convince ourselves by experiment that stack reification really does
take linear time. Figure~\ref{fig:benchmark_program} shows a Haskell
program that runs the exact same pure computation twice, the first time
it runs the computation with a small execution stack and the second time
a lot of frames have been pushed on the execution stack. We also run the
program twice, the first time with stack traces enabled and the
second time with stack traces disabled:

\begin{figure}
\begin{mdframed}
  \begin{minted}[gobble=4]{haskell}
    import Control.Exception ( catch
                             , SomeException (SomeException)
                             , evaluate)
    import System.IO.Unsafe (unsafePerformIO)
    import Data.List (foldl')
    import System.CPUTime (getCPUTime)

    main :: IO()
    main = do
        evaluate (stackBuilder 1) -- For fairness with CAFs
        putStrLn $ "Takes " ++ show fast ++ " ms with small stack"
        putStrLn $ "Takes " ++ show slow ++ " ms with large stack"
      where
        fast = stackBuilder 0
        slow = stackBuilder 1000

    getMilliSeconds :: IO Integer
    getMilliSeconds = fmap (`div` 1000000000) getCPUTime

    timeIt :: Integer -> IO Integer
    timeIt x = do t0 <- getMilliSeconds
                  evaluate x
                  t <- getMilliSeconds
                  return (t - t0)

    -- Returns the time it takes to evaluate pureFunction
    stackBuilder :: Integer -> Integer
    stackBuilder x | x < 1 = unsafePerformIO (timeIt (pureFunction x))
    stackBuilder x         = x - x + stackBuilder (x-1) -- Push frames

    -- It has to take an argument to not become a thunk
    pureFunction :: Integer -> Integer
    pureFunction zero =
        foldl' (+) 0 (map stupidFunction [1..(zero + 1000000)])

    stupidFunction :: Integer -> Integer
    stupidFunction x = unsafePerformIO (action `catch` handler)
      where
        action = evaluate (5 `div` 0)
        handler (SomeException _) = return x
  \end{minted}
  \caption{A Haskell program we use for benchmarking purposes.}
  \label{fig:benchmark_program}
\end{mdframed}
\end{figure}

\begin{minted}[gobble=2, samepage=true]{bash}
  $ ./Benchmark +RTS --stack-trace -RTS
  Takes 332 ms with small stack
  Takes 18344 ms with large stack

  $ ./Benchmark
  Takes 175 ms with small stack
  Takes 178 ms with large stack
\end{minted}

The results clearly match the hypothesis. If we do not reify the stack
at all like in the second program run. The two functions are equally
fast. But if we do one stack reification per exception (as happens in
\texttt{stupidFunction}), the program gets slower as seen from the
results in the first program run. We can also see that the program
gets \emph{much} slower if the execution stack is already big. This
is considered to be a serious problem because \texttt{pureFunction}
is not supposed to depend on its environment. 
The tests were run on a Ubuntu 13.10 laptop, Intel(R) Core(TM) i7-4800MQ CPU at
2.70GHz with GHC revision \cite{github_benchmarking_revision}, we compile with
\texttt{-rtsopts -g --make} as flags to \texttt{ghc}.

As we just saw, reifying the stack is costly and we will dedicate this whole
section for finding more efficient alternatives. The two natural solutions are to
either reify the stack \emph{conditionally} or
to reify it \emph{lazily}. When reifying the stack conditionally, we only reify the stack
when we know for sure that we are going to print the stack. Subsection~\ref{reify_static_analysis} looks at solutions in this category.
The second approach would be to let the stack value be lazily evaluated. If creating the thunk for
the execution stack value could be done in constant time, we have a
satisfactory solution. Subsection~\ref{sec:stack_freezing}
and~\ref{sec:chunked_reifying} are implementation ideas for lazy stack values.
But first we take a look at a simple and
unsophisticated solution in subsection~\ref{sec:constant_frames}.

\subsection{Reifying a constant number of frames} \label{sec:constant_frames}

A very simple solution would be to just reify a constant number of
frames. The time complexity for any reification would then be constant.
The exact number of frames could be specified through a run-time parameter
or from Haskell-land. There is also a few other benefits with this
approach, for one there would be no user-interface issues with too
long stack traces being printed to the screen, second, as we saw from subsection~\ref{sec:update_frames}, the top of the stack is less likely to have
been blackholed. The drawback is that the stack trace could
be too truncated to be useful.

To verify our speculations, we run the benchmarking program from
figure~\ref{fig:benchmark_program} with the frame size limits $10$ and $500$:

\begin{minted}[gobble=2, samepage=true]{bash}
  $ ./Benchmark +RTS --stack-trace --reify-x-frames 10 -RTS
  Takes 232 ms with small stack
  Takes 236 ms with large stack

  $ ./Benchmark +RTS --stack-trace --reify-x-frames 500 -RTS
  Takes 315 ms with small stack
  Takes 4120 ms with large stack
\end{minted}

This makes sense. In the case with a cap on reifying $10$ frames, we see
that both computations take about the same time. It also takes more time
than when not reifying a stack at all (since $10 > 0$), but it takes
less time than when the stack is unbounded, since $10$ is less than the
number of frames on the stack. For the case when reifying $500$ frames,
we see that the computation on a small stack is taking as much time as
it did then we reified the whole stack, this is because the whole stack
is less than $500$ frames. The computation done with the larger stack takes
about one fourth of what it was when we reified the whole stack, this is because
we call \texttt{stackBuilder} $1000$ times, and each time it pushes two frames
(one addition, one subtraction).

\subsection{Static analysis} \label{reify_static_analysis}

By analyzing Haskell source code at compile time we can get
information that could help decide whether we should reify the stack or
not. For instance, when the compiler generate code for the \texttt{throw}
primitive operation, the \emph{compiler} could choose if a stack should be reified or
not at this particular usage of \texttt{throw}. Another idea is to do
static analysis on the uses of the \texttt{catch} primitive, one could mark
the catch frames that are using the stack value and choose at
run-time to reify the stack if the catch frame needs it.

To do static analysis on the uses of \texttt{catch} has no runtime cost, but
there many problems. When the execution stack contains a series of catch
statements and if the topmost catch frame indicates that it doesn't need the stack,
it will not be possible to rethrow the stack trace to the second catch frame which
might need it. Section~\ref{sec:rethrowing_the_stack} in the next chapter
will discuss semantics of rethrowing in detail.

Another serious obstacle with static analysis is that the code
generation phase would require a lot more context to decide what kind
of catch or throw site it is. Catch and throw sites have to generate
code defensively (like with all compiler optimizations). While code from
remote libraries must be compiled defensively as one would expect, the
biggest issue is thunks! Since thunks passed as arguments can contain
arbitrary code, even code that throws, the compiler have to assume the
worst.

We have not found any sensible way to do static analysis that prove that
a throw at a throw site don't need to reify the stack. We reach the same
conclusion when doing analysis on catch sites. Even if a handler can be
proven to not use stack traces, we might still need reify the stack in
order to pass it down to the next handler.

\subsection{Stack freezing} \label{sec:stack_freezing}

One issue with lazy reification is that the stack is a mutable data
structure, so it is not enough to have a reference to the
topmost chunk. The issue of mutability goes away if we make the stack structure
immutable when reifying\footnote{So far, all stack reifications have been for
  the purpose of creating a stack value. The name reification has been used
  since all methods discussed until now have been about copying the essentials
  of the stack and store it in changed format. To avoid too much terminology in
  this paper, we call \emph{all} methods of creating a stack value for
  ``reification''.}, we call that \emph{stack freezing}. To freeze the whole stack
would mean to freeze each individual stack chunk. A frozen chunk's
content should be regarded as read-only and the reference to the top
of the stack should not change either. It turns out that thanks to
already existing machinery, much is already implemented.

Freezing a particular chunk (a \texttt{StgStack} value) would be
trivial by just setting the chunk's \texttt{stack\_size} to be zero
and saving the \texttt{sp}-value to another field. By setting
the \texttt{stack\_size} to zero, the overflow check described in
section~\ref{sec:structure_of_stack} would automatically kick in as a
copy-on-write mechanism. Since we saved the \texttt{sp} value at the time of the reification,
the \texttt{sp} value can change, which means that the stack chunk
can still safely shrink. The stack value itself would be a value with
a reference to the stack chunk that was current at the time of the
exception, as seen in figure~\ref{fig:freezing}.

\begin{figure}
\begin{mdframed}
  \includegraphics[width=5.5in]{build/fig/freezing}
  \caption{Illustration of stack freezing. The frames with \texttt{size = 0}
    are the frames that have been frozen by the run time system when a crash
    was detected.}
  \label{fig:freezing}
\end{mdframed}
\end{figure}

So far this solution is linear, since freezing the whole stack would mean
to traverse through each chunk and freezing them. The number of chunks is
linear in the size of the stack. Luckily, we can get away with freezing
the whole stack by only freezing the chunk where the handler lies.
Because when we get an underflow, control is passed to the RTS
\cite{github_underflow_frame}. From the RTS code we could freeze
the next chunk of the underflow frame, which can be thought of as freezing lazily.

There are multiple drawbacks of freezing the stack. When doing a garbage
collection, a stack value has a reference to the old stack and it would
be wasteful to retain the payloads and all their references if the
eventual printed stack trace is only based on the info pointers. Another
drawback is that it is not clear when (if ever) to \emph{thaw} the
stack. Thawing would be the process of undoing the freezing of the
stack. Freezing the stack and the chunk is very cheap, but a frozen chunk
will cause overflows on every push. Thawing can be done in
constant time by just restoring the \texttt{stack\_size} of the current
stack chunk. Thawing would ideally happen when the last stack value
become unreachable from Haskell code. Subsection~\ref{sec:stack_thawing}
will look at how the RTS can know when to initiate the thawing.

The overall advantage of stack freezing is that we would never actually
need to create another array, so then all payloads are still accessible.
If the payloads are available from the stack value in the handler,
the optimizations from section~\ref{sec:frames_of_interest} could be
configurable from code in Haskell-land. Even better, one day the stack
traces could utilize the stack frames even more by for example printing
the values of the free variables in a case continuation.

\subsection{Stack thawing} \label{sec:stack_thawing}

When we defined freezing and thawing in subsection~\ref{sec:stack_freezing} we
said that both freezing and thawing were quite simple operations. However, we still need
some sort a destructor mechanism for heap objects since it is only safe to thaw the
\texttt{StgStack} chunk when it is not used anymore.  The RTS in
GHC does support \emph{weak pointers}. Unlike regular pointers, weak pointers
do not maintain liveliness for the object it is pointing at. During garbage
collection, The weak pointers are visited after all other live pointers have
been visited, in that way it is possible to see if a heap object have died
since the last garbage collection. When a dead weak pointer is visited, the pointer is removed and a
\emph{finalizer} is run. It is tempting to let thawing be the finalizer of
stack values, but there could be values that are
created from the stack value that relies on the stack being left intact.  For
example, we could have that the stack is exposed to the programmer as a Haskell
list.  If any of the list values are alive the stack must not thaw as the data source
for the list (the stack) can then be overwritten. One way to ensure that the
finalizer is not run prematurely is for every heap object that depends on the
stack being alive to have a pointer to the stack value. This way the stack
value will be alive as long as its dependents are alive.

Another design question is if the finalizer should thaw the stack. The
stack is owned by a particular capability, but the finalizer can be run by
any capability doing the garbage collection. We propose that thawing only
should happen on stack underflows, because then we are certain that the running
thread is thawing its own stack and we do not need to worry about the overhead of
locks and there being race conditions. But then what would the finalizer do? How would a underflowing
stack chunk know if it should thaw the succeeding chunk? We propose a scheme like in figure~\ref{fig:thawing}:
Every \texttt{StgStack} chunk has a linked list of subscribers that require the
stack to be kept frozen (in the figure we just have one pointer and do
not show the linked list structure). The only weak pointer is pointing at the stack value
and its finalizer is to mark the subscriber as dead. When a chunk underflows it
will traverse its linked list and remove the entries pointing to dead
subscribers. The next chunk will then be be frozen if and only if the subscriber list is nonempty.
The list will then be inherited by the succeeding chunk.

\begin{figure}
\begin{mdframed}
  \includegraphics[width=5in]{build/fig/thawing}
  \caption{A scheme for stack thawing without locks.}
  \label{fig:thawing}
\end{mdframed}
\end{figure}

One alternative for a linked list to subscribers could be to have a counter
containing the number of subscribers. But that would require synchronization
primitives.

\subsection{Chunked reifying} \label{sec:chunked_reifying}

Chunked reifying of the stack is another idea about doing the stack
reification lazily. In chunked reification we do not freeze the stack, instead, we observe
that the stack is \emph{almost} immutable already! The stack can only
modify itself frame by frame unlike an array which can access any
element. This is a powerful property since this means that the stack is
almost immutable. Like for stack freezing we want to have some sort of
copy-on-write mechanism and we will save frames once they are popped from
the stack. It is too expensive to copy over one frame each time the
stack pointer goes below its yet lowest point since the reification. Instead we
can copy over stack frames in chunks.
Luckily, the execution stack implemented in GHC is already chunked
and we could decide to reify the next chunk on each underflow.
One very useful trick is to save the reified stack chunk in the chunk that
underflowed and not in its successor. It might feel more natural at first to
store the reified frames for chunk $A$ in chunk $A$, but if chunk $A$ overflows
again and does another stack reification, then $A$ has to hold two
\emph{different} reified chunks and the stack values would not know which one
belongs to them. We avoid this problem by storing the chunk in the overflowing chunk.

Like with the approach of stack freezing, it would improve performance to do
something similar to thawing the stack when stack values become inaccessible.
Even if chunked reification does not freeze the stack it must be able to tell
stack parents that they do not need to reify on underflows when no stack value
is alive anymore.  The finalizer scheme could be the same as the one described in
subsection~\ref{sec:stack_thawing}.

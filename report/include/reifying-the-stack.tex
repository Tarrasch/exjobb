\chapter{Reifying the Stack}

In the previous chapter we looked at how the stack works, in its
last section we saw the process of reifying the stack as it was
implemented in Peter Wortmann's demonstration in August 2013
\cite{stack_traces_ticket}. This chapter will be dedicated to examine
the when and how of stack reification in detail. By looking critically
of the prototype, we find room both for improvement and discussion.

In section \ref{sec:frames_of_interest} we see that the stack traces can become more
readable by using the extra information in the payloads of each stack
frame. Section \ref{sec:lazy_reification} deals with the issue of wasting
resources when reifying the stack without ever using it.

\section{Frames of interest} \label{sec:frames_of_interest}

Figure \ref{fig:traces} showed that the stack trace from Peter Wortmann's
demonstration is very far from the ideal stack trace.  Worse, the only
information we have to work with is the execution stack, remember, we maintain no
explicit call stack for performance reasons. Still, the stack can become
clearer by using the payload of the stack frames. We reproduce the stack trace here again:

\begin{minted}[gobble=4]{text}
     0: stg_bh_upd_frame_ret
     1: stg_bh_upd_frame_ret
     2: stg_bh_upd_frame_ret
     3: showSignedInt
     4: stg_upd_frame_ret
     5: writeBlocks
     6: stg_ap_v_ret
     7: bindIO
     8: bindIO
     9: bindIO
    10: bindIO
    11: stg_catch_frame_ret
\end{minted}


\subsection{Update frames} \label{sec:update_frames}

The actual code that is executing under the wing of an update frame
could be anything, to see update frames on the stack is hardly helpful.
But we know that the field of the update frame is a reference to the
heap object the update frame is going to update, the \emph{updatee}. The updatee's
info table contains the code that we're currently running, so we instead
copy that data into our array. See figure \ref{fig:reify_update_frame}.

\begin{figure}
\begin{mdframed}
  \includegraphics[width=5in]{build/fig/reify_update_frame}
  \caption{An improved way of reifying update frames. The old method is
  showed in the dashed line.}
  \label{fig:reify_update_frame}
\end{mdframed}
\end{figure}

Frame 0,1,2 and 4 are all update frames. There is exactly three kinds
of update frames \cite{github_updates_cmm}:

\begin{itemize}
  \item
    \texttt{stg\_bh\_upd\_frame} is the update frame used for global
      thunks. Global thunks are better known as CAFs.
  \item
    \texttt{stg\_upd\_frame} is the update frame used for local
  thunks. \item \texttt{stg\_marked\_upd\_frame} is created
  by overwriting one of the other two kinds of frames, it
  happens during garbage collection and this phase is named
  \emph{blackholing}. Blackholing converts any update frame to a
  marked update frame \cite{github_overwrite_update_frame} and
  overwrites the updatee's info pointer to point to a "black hole"
  \cite{github_overwrite_blackhole}.
\end{itemize}

Unfortunately, only the \texttt{stg\_upd\_frame}'s updatee point
at interesting code and not a black hole. The marked update frame
always gets its updatee's info pointer overwritten to a black
hole \cite{github_overwrite_blackhole} and the update frame
for global thunks' updatee point at a black hole to begin with
\cite{github_set_hdr_caf_blackhole}.

\subsubsection{Retaining code reference on blackholing}

The bad effect of blackholing from a stack trace perspective is
that thunks loose the reference to the code they're executing. So instead
of being able to print something useful, like  \texttt{... 4: print ...},
we have to print \texttt{... 4: stg\_upd\_frame\_ret ...}.

Unfortunately, blackholing is not optional
\cite{github_blackholing_not_optional}. It is however possible to retain
the code reference by just adding a field for thunks and copying over
the code pointer before overwriting. This will have a runtime cost of
course.


\section{Lazy reification} \label{sec:lazy_reification}

The reification done when executing the program from figure
\ref{fig:sample_program} need to happen at the time of the crash and
not in the handler. When control have been passed to the handler the
execution stack have been unrolled already, so the stack is unaccessible
and maybe even overwritten. At the same time, there is a cost associated
with reifying the stack and the cost is growing linearly with the size
of the execution stack. So when we \emph{know} that we're going to print
out the stack, it is acceptable to have the additional linear cost of
reifying the stack, but we shouldn't tolerate the cost when we are not
using the reified stack value. One consequence of always reifying the
stack is that functions that use \texttt{throw} for control flow will
become slower depending on how big the stack is when they're called.

The discussion in the previous paragraph have one natural solution, to
let the stack value to be \emph{lazily evaluated}. Then we would only
reify the execution stack when we need to. If creating the thunk for
the execution stack value could be done in constant time, we obtain the
ideal solution per the previous discussion. In this section we look at
various solutions to solve the problem that functions run-time depending
on the size of the stack. Some solutions will not need to implement
lazy stack values.

\subsection{Reifying a constant number of frames}

A very simple solution would be to just reify a constant number of
frames. The time complexity for any reification would then be constant.
The exact number of frames could be set through a run-time flag. There
is also a few other free benefits with this approach, for one there
would be no user-interface issues with too long stack traces being
printed out, second, as we saw from section \ref{sec:update_frames},
the top of the stack is less likely to have been blackholed and would
contain more frames with useful information. As a downside, the stack
trace could be too truncated to be useful.

\subsection{Static analysis}

By analyzing Haskell source code at compile time we can get
information that could help decide whether we should reify the stack or
not. For instance, whenever we generate the code for the \texttt{throw}
primop, we could choose at compile time if a stack should be reified or
not at this particular usage of \texttt{throw}. Another idea is to do
static analysis on the uses of the \texttt{catch} primop, one could mark
the catch frames that are using the stack value or not and choose at
run-time to reify the stack value only if the catch frame needs it.

The last out of these two ideas sounds promising, it has no runtime cost
and sounds easy to implement, but there is a problem. When the execution
stack contains a series of catch statements. If the topmost catch frame
indicates that it doesn't need the stack it will not be able to rethrow
the stack trace to the second catch frame which might need it. Section
TODO from the next chapter will discuss semantics of rethrowing in
detail.

\subsection{Stack freezing}

One issue with lazy reification is that the stack is a mutable data
structure, so it's not enough to just remember a reference to the
topmost frame. But if we implement a way to "freeze" the stack, these
issues would disappear. Freezing a particular chunk (a \texttt{StgStack}
value) would be trivial by just setting the stack size to be zero. The
overflow mechanism described in section \ref{sec:structure_of_stack}
would automatically kick in. The stack value would then be a value with
a reference to the stack chunk that was current at the time of the
exception. As seen in figure \ref{fig:freezing}.

\begin{figure}
\begin{mdframed}
  \includegraphics[width=5.5in]{build/fig/freezing}
  \caption{Illustration of stack freezing. The frames with \texttt{size = 0}
    are the frames that have been frozen by the run time system when a crash
    was detected.}
  \label{fig:freezing}
\end{mdframed}
\end{figure}

One advantage of this idea is that we would never actually need to
do any reification in the sense that we create another array, so
then all payloads are still accessible. Allowing for the stack trace
optimizations from section \ref{sec:frames_of_interest} to be choosen much later, for example
when we want to print the stack. A far fetched idea could even involve
printing out variables (the fields).

\subsection{Chunked reifying}

Chunked reifying of the stack is another idea about doing the stack
reification lazily. In chunked reification we do not however do the
perhaps costly operation of freezing the stack, instead, we observe
that the stack is \emph{almost} immutable already! The stack can only
modify itself frame by frame unlike an array which can access any
element. This is a powerful property since this means that the stack is
almost immutable. The idea of chunked reification is that when the stack
pointer moves down towards the bottom of the stack, we start
copying over frames before they get overwritten. 

Luckily, the execution stack as implemented in GHC is already chunked
(section \ref{sec:structure_of_stack}). Even better, when we get an
underflow, control is passed to the rts \cite{github_underflow_frame}.
From the rts code we could reify the chunk that the underflow frame is
referencing. Where exactly the reified chunk should be stored is an
interesting issue, it could for example be in the current underflow
frame or inside the lazy stack value somehow.

Like with the approach of stack freezing, it would help performance
to thaw the stack if all stack values become unaccessible. While chunked
reification doesn't freeze the stack it must be able to tell stack
parents that they don't need to reifying the stack when no stack value
is alive anymore.

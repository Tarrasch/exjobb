\chapter{Introduction}
% \lettrine[lines=4, loversize=-0.1, lraise=0.1]{L}{orem ipsum dolor} Blah blah blah % fixme: sätt tillbaka i slutet

Part of software development is debugging. Debugging is the activity of
diagnosing flawed software, in practice this means finding programming mistakes
in the program source code. Debugging is usually initiated when the running
software behaves unexpectedly, like crashing. When a program crashes, it'll be
required for a programmer to diagnose it in order to find the root cause and
correct it. As software systems become increasingly complex, they will grow in
code size and the debugging phase becomes a more involved process.

From an economical perspective, a lack of funding or commercial success can
stagnate the growth of a software project. But from a technological perspective, a
software project stagnates due to lack of technology that scales. There are
software tools that keeps growing software systems to remain manageable, even for
hundreads of developers and millions of lines of code. These tools include
development environments, version control and programming language features
like interfaces. All of which are part of a programmers day to day work.
Another set of tools that become essential as software systems grows are those
that ease debugging.

This work implements and analyzes an implementation of \emph{stack traces} for
the programming language \emph{Haskell}. Stack traces is a language feature
that prints out extra context on a program crash, making the task of debugging
the software much easier. Haskell is both a research language
\cite{haskell_org_research_papers}
\cite{dagit_getting_started_with_ghc_hacking} and used in industry
\cite{haskell_in_industry} \cite{fpcomplete_case_studies}.


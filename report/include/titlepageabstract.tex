% Chalmers title page
\hypersetup{pageanchor=false} % http://alturl.com/ytpfp
\begin{titlepage}

\AddToShipoutPicture{\backgroundpic{-4}{56.7}{fig/auxiliary/frontpage}}
\mbox{}
\vfill
\addtolength{\voffset}{2cm}
\begin{flushleft}
	{\noindent {\Huge Stack Traces in Haskell} \\[0.5cm]
	\emph{\Large Master's Thesis in Algorithms, Languages and Logic} \\[.8cm]

	{\huge ARASH ROUHANI}\\[.8cm]

	{\Large Department of Computer Science and Engineering \\
	\textsc{Chalmers University of Technology} \\
	Gothenburg, Sweden 2014 \\
	Master's Thesis 2014:??\\
	}
	}
% TODO: fixa ?? ovan
\end{flushleft}

\end{titlepage}
\ClearShipoutPicture
% End Chalmers title page

\pagestyle{empty}
\newpage
\clearpage
\mbox{}
\newpage
\clearpage
\thispagestyle{empty}

\begin{abstract}

  This thesis presents ideas for how to implement Stack Traces
  for the Glasgow Haskell Compiler. The goal is to come up with
  an implementation with such small overhead that organizations do not
  hesitate to use it for their binaries running in production. Since the
  implementation is aiming for efficiency it will be heavily
  tied to only GHC.
  This work has been made possible thanks to a very
  recent contribution \cite{DBLP:conf/haskell/WortmannD13} that
  implements debug data for binaries compiled with GHC. Thanks to that
  contribution, this thesis can almost entirely focus on managing the GHC
  stack.
  Three different designs of stack values is presented, they allow creation
  in constant time and we implement one of these designs.
  The overhead of these designs can be kept small by utilizing laziness and the
  special linked list structure of the GHC stack.
  The other contribution is the work on the Haskell API that is exposed to
  programmers. We have implemented an API where the Haskell programmer can
  create the stack value at will and examine its content.  Different ways of incorporating
  stack traces into the catching and throwing mechanism have been
  analyzed and we have found a rethrowing semantics for Haskell that is
  backwards compatible, convenient to use and easy to implement in GHC.
  If the most promising design for stack values works out, stack values will be
  exposed as first class values in Haskell.

\end{abstract}

\newpage
\clearpage
\mbox{}
\newpage
\clearpage
\thispagestyle{empty}
% TODO: uncomment for final non-draft
% \section*{Acknowledgements}
% Thanks allyaall. \\[1cm]

% \hfill Arash Rouhani, Somewheretown 11/9/11
% \newpage
% \clearpage
% \mbox{}

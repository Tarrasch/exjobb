% Chalmers title page
\hypersetup{pageanchor=false} % http://alturl.com/ytpfp
\begin{titlepage}

\AddToShipoutPicture{\backgroundpic{-4}{56.7}{fig/auxiliary/frontpage}}
\mbox{}
\vfill
\addtolength{\voffset}{2cm}
\begin{flushleft}
	{\noindent {\Huge Stack Traces in Haskell} \\[0.8cm]
	\emph{\Large Master of Science Thesis} \\[2.5cm]

	{\huge ARASH ROUHANI}\\[3.8cm]

	{\Large Chalmers University of Technology \\
  Department of Computer Science and Engineering \\
  Göteborg, Sweden, March 2014
	}
	}
\end{flushleft}

\end{titlepage}
\ClearShipoutPicture
% End Chalmers title page

\pagestyle{empty}
\newpage
\clearpage
The Author grants to Chalmers University of Technology and University of
Gothenburg the non-exclusive right to publish the Work electronically
and in a non-commercial purpose make it accessible on the Internet.

The Author warrants that he/she is the author to the Work, and warrants
that the Work does not contain text, pictures or other material that
violates copyright law.

The Author shall, when transferring the rights of the Work to a third
party (for example a publisher or a company), acknowledge the third
party about this agreement. If the Author has signed a copyright
agreement with a third party regarding the Work, the Author warrants
hereby that he/she has obtained any necessary permission from this
third party to let Chalmers University of Technology and University of
Gothenburg store the Work electronically and make it accessible on the
Internet. \\
\\
\\
\\
\\
\\
\\
\\
\\
\\
\\
\\
\\
\\
Stack Traces for Haskell \\
\\
A. ROUHANI \\
\\
\copyright A. ROUHANI, March 2014. \\
\\
Examiner: J. SVENNINGSSON \\
\\
Chalmers University of Technology \\
University of Gothenburg \\
Department of Computer Science and Engineering \\
SE-412 96 Göteborg \\
Sweden \\
Telephone + 46 (0)31-772 1000 \\
 \\
Department of Computer Science and Engineering \\
 \\
Department of Computer Science and Engineering \\
Göteborg, Sweden March 2014 \\

% \mbox{}
% \newpage
% \clearpage
\thispagestyle{empty}

\begin{abstract}

  This thesis presents ideas for how to implement Stack Traces
  for the Glasgow Haskell Compiler. The goal is to come up with
  an implementation with such small overhead that organizations do not
  hesitate to use it for their binaries running in production. Since the
  implementation is aiming for efficiency it will be heavily
  tied to only GHC.
  This work has been made possible thanks to a very
  recent contribution \cite{DBLP:conf/haskell/WortmannD13} that
  implements debug data for binaries compiled with GHC. Thanks to that
  contribution, this thesis can almost entirely focus on managing the GHC
  stack.
  Three different designs of stack values is presented, they allow creation
  in constant time and we implement one of these designs.
  The overhead of these designs can be kept small by utilizing laziness and the
  special linked list structure of the GHC stack.
  The other contribution is the work on the Haskell API that is exposed to
  programmers. We have implemented an API where the Haskell programmer can
  create the stack value at will and examine its content.  Different ways of incorporating
  stack traces into the catching and throwing mechanism have been
  analyzed and we have found a rethrowing semantics for Haskell that is
  backwards compatible, convenient to use and easy to implement in GHC.
  The design in this paper allows stack values to be first class values.

\end{abstract}

\newpage
\clearpage
\mbox{}
\newpage
\clearpage
\thispagestyle{empty}
\section*{Acknowledgements}

I would like to thank my supervisor Josef and my opponent Dima at
Chalmers for always being available and helpful. This work would have
never been possible without Peter, who helped me every time I got stuck.
Simon have been guiding and reviewing my work and was kind and arranged
for me to meet Peter in person at Facebook's office in London. I am also
very grateful for Pepe's feedback and review of my thesis. Last but
definitely not least, I would like to thank my two parents and my two
brothers, who have been there for me throughout my whole life. \\[1cm]

\hfill Arash Rouhani, Göteborg, Sweden, March 2014

\newpage
\clearpage
\mbox{}

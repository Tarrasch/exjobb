\chapter{Related work}

To programmers outside of the Haskell community, it could sound
surprising that a mature language like Haskell doesn't support stack
traces. This might raise the following questions:

\begin{itemize}
  \itemsep1pt\parskip0pt\parsep0pt
  \item
    Since stack traces are difficult, what other means of debugging are
    there?
  \item
    Are stack traces in Haskell really necessary?
  \item
    Is there at least any inefficient way to get stack traces?
  \item
    How close is the Haskell community in solving stack traces?
\end{itemize}

The overall structure of this chapter is that we answer the questions
by looking at related work. Most of the
related work is recent, usually less than 5 years from the time of
publishing of this work. The first question is answered in section
\ref{sec:debugging_haskell}. Section \ref{sec:avoiding_crashing}
shows that Haskell is a language producing robust programs, which causes
crash reports like stack traces less likely to get reported.
The third question is answered in section
\ref{sec:overhead_full} which shows many working implementations of
stack traces, all of which have significant overhead. The last question
is answered in section \ref{sec:recent_work}.

\section{Debugging Haskell} \label{sec:debugging_haskell}

Examining stack traces falls in the category of debugging. Programmers
examine stack traces printed from a handled exception or a
crash. The amount of time the program runs before it crashes can be
anywhere between a few nanoseconds to many years. Stack traces are most
valuable for programs that crash unexpectedly after a long time of
stable execution, because it might be hard to reproduce the error in
order to diagnose it. Ideally, the stack trace aids the programmer in
writing a minimal reproducible test case that exercises the original
bug. Once the programmer have an easy to reproduce bug, they look for
tools that help them better understand why the bug is happening. In this
subsection We'll look at existing tools for GHC that lets the programmer
step through the programs execution and even print variables. Neither of
which the stack trace implementation in this paper can provide.

\subsection{GCHi Debugger}

GHC comes with its own interactive \emph{read evaluate print loop} (REPL)
which have an in-built debugger. It's rich in features, supporting break
points, single-stepping, breaking on crashes, a "tracing mode" and even
variable inspection. The implementation works only with interpreted
code. \cite{ghci_debugger} So there will be significant overhead both
from the fact that the code is interpreted and that the debugger is
running.


\subsection{ghc-vis}

Debuggers are a view into the otherwise opaque executing program. The
GHCi debugger interface is text based, the programmer enters a command
to the debugger and it responds in text.  ghc-vis on the other hand is a
graphical debugger.  It allows users to visualize variables and interact
with them with the mouse pointer. For example, when clicking on an yet
unevaluated expression (remember, Haskell is a lazy language) it will
evaluate the expression.  Making it great for stepping through your
program without losing the big picture.  ghc-vis is hooking itself into
the program by running a thread inside of ghci. So it will only work
with ghci.  \cite{thesisFelsingBA}

% TODO behövs mer här?

\section{Avoiding Crashing} \label{sec:avoiding_crashing}

If a program never crashes,
it will not matter if our language prints stack traces or not. Provably never-crashing programs is a research area, sometimes called formal
verification. There are many approaches to formal verification. One
can statically analyze C-programs \cite{ckl2004},   % TODO source:
use finite automata % TODO source:
or formal grammars \cite{dantam2013motion} % TODO source
and use type system tricks. % TODO source (hittade inget bra, skita i bara?)
In this section We'll take a deeper look at some methods that
relates to verifying Haskell code in particular. Note that none
of these methods will be perfect. If they were perfect, we would not
need stack traces in the first place.

\subsection{Catch}

Catch is a static analyzer for Haskell. It can detect if a pattern-matching is
sufficiently covering, even if the cases aren't collectively exhaustive. Figure
\ref{fig:catch_example} shows a function where the pattern match isn't exhaustive but
sufficiently so.
Catch can prove that such a pattern
matching is safe by doing flow analysis and ruling out impossible
patterns for the scrutiny (the expression that we \texttt{case} on). % TODO ta bort jargong?
This eliminates the need for the human programmer to manually check what can be
automatically proven. \cite{mitchell:catch_2008_9_25}

\begin{figure}
\begin{mdframed}
      \begin{minted}{haskell}
safeFunction = nonExhaustivePatterns False
  where
    nonExhaustivePatterns False = 42
      -- NOTE: No pattern for True
      \end{minted}
      \caption{A safe function even though the non-exhaustive matching. A
        totality Checker like Catch can ensure that it's safe.}
      \label{fig:catch_example}
\end{mdframed}
\end{figure}

\subsection{Agda}

TODO: Prata med Josef

\section{Overhead-full stack traces} \label{sec:overhead_full}

There are already many successful stack trace implementations in
Haskell. Unfortunately tough, they all have a significant overhead.
In this section we will look at previous work about stack traces for
Haskell who all essentially work by maintaining a separate structure
representing the stack that's passed along during execution. Unlike this
work they don't use the existing execution stack.

\subsection{The general idea}

\begin{figure}
\begin{mdframed}
        \begin{subfigure}[t]{0.5\textwidth}
            \begin{minted}{haskell}
main = print (f 100)

f :: Int -> Int
f x = g (5*x)

g :: Int -> Int
g 7 = error "Bang"
g x = 100 * x
            \end{minted}
            \caption{Original program}
        \end{subfigure}
        ~ %add desired spacing between images, e. g. ~, \quad, \qquad etc.
          %(or a blank line to force the subfigure onto a new line)
        \begin{subfigure}[t]{0.5\textwidth}
          \begin{minted}{haskell}
main = print (f stack' 100)
  where
    stack' = "main \n"

f :: String -> Int -> Int
f stack x = g stack' (5*x)
  where
    stack' = "f (case 1)\n" ++ stack

g :: String -> Int -> Int
g stack 7 = error ("Bang" ++ stack')
  where
    stack' = "g (case 1)\n" ++ stack
g stack x = 100 * x
  where
    stack' = "g (case 2)\n" ++ stack
          \end{minted}
          \caption{Transformed program}
          % \label{fig:tiger}
        \end{subfigure}
        \caption{An example of how a Haskell program can be transformed to one
          that will print stack traces on errors. The syntax `\texttt{str1 ++
            str2}' is string concatenation.
        }\label{fig:transformation}
\end{mdframed}
\end{figure}

Stack traces can be achieved by doing some methodological source level
transformations. Figure \ref{fig:transformation} shows a program transformed
into one producing stack traces on calls to \texttt{error}. This transformation is essentially:

\begin{itemize}
\itemsep1pt\parskip0pt\parsep0pt
\item
  Changing all top level functions to take one additional string
  argument. Except for the program entry-point \texttt{main}.
\item
  Transform all equations to define the new call stack \texttt{stack'} and
  pass it as the first argument to all calls of top level functions.
\item
  Transform all calls to \texttt{error} to also print out the call stack.
\end{itemize}

This transformation is similar to
\cite{ghc_wiki_explicitcs_transformation}  and a complete
source-to-source implementation called \emph{hat} exists.
\cite{hat_website}.  While the idea is simple there are many
complications and drawbacks as noted by other implementors.

\subsection{StackTrace}

Allwood et al implemented a Intermediate Representation (IR) transformation
pass called \emph{StackTrace}. It's operating on an IR called GHC Core which
is like a small subset of Haskell. So its implementation will do something
similar to what figure \ref{fig:transformation} illustrates.

Among it's complications are the
handling of higher order functions, linking with code that doesn't have stack
traces and an efficient non-naive implementation of the
passed along stack. \cite{FindingTheNeedle2009}
Functional programming in particular relies on efficient tail call
optimizations, which
requires the passed around call stack to efficiently handle this.

% TODO, Detta (ContMarks) passar inte in alltså, iom allt annat har med just
% Haskell att göra plus att det ska vara om overhead-full stack traces.
% detta borde ju bli overhead-full men svårt att bara påstå det.
% \subsection{Continuation Marks}

% The other dimensionality of complications is the implementation
% complexity.  There are other debugging features besides stack traces a
% compiler would like to support, and the more features the bigger blowup
% in the complexity of the compiler itself.  Clements disseratation shows
% a framework called \emph{continuation marks} which allows for a generic
% way to create addons to the language \cite{clements_dissertation2005}.
% In his disseratation, he uses continuation marks as a common ground of
% implementation for stack traces, code stepping in debugging and aspect
% oriented programming.  Since GHC does not have continuation marks
% implemented, it's not a possible starting point for implementing stack
% traces.

\subsection{Stack traces with profiling}

A mature and stable implementation of overhead-full stack traces for
Haskell is present in GHC since GHC 7.4.1 (February 2012). No paper have
been produced from this effort. But a talk were given at Haskell
Implementors Workshop in 2012 \cite{HIW2012Programme}.
The implementation is only working in conjunction with the profiling
mode of GHC. In Profiling mode the execution of programs can expect to
be twice as slow as their plain counterparts.

\section{Recent work on Stack Traces for Haskell} \label{sec:recent_work}

Around the time when this thesis started, Peter Wortmann, a
PhD candidate at University of Leeds showed a proof of concept
stack trace in Haskell that was based on the execution stack.
\cite{stack_traces_ticket} Peter had been working on non intrusive
profiling for GHC. To accomplish this, he had developed a theory of
causality of computations in Haskell, in particular his work extended to
optimized code too. \cite{DBLP:conf/haskell/WortmannD13} To do profiling
he needed to relate instruction pointers to the corresponding source
code. He added source code annotations (see figure TODO) that propagated
through the pipeline of IRs and optimizations and finally emitted DWARF
debugging data. As of his work GHC now integrates DWARF, making a
stack trace implementation to be low hanging fruit, enabling the quite
sizeable problem of stack traces to be worked on during the limited
scope of a master's thesis.

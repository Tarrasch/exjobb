\section{Background}

% TODO: text här? Upplägg?

\subsection{Stack traces}

When a computer program crashes, the runtime of some programming languages
gives some context to where in the code the program crashed.
Typically, a \emph{stack trace} is printed. A stack trace is the listing
of the functions that have called each other and have not exited yet, so
they have all been part of the crash. The first function in the stack
trace is always the program entry point, the last function is where the
crash actually occurred.

% TODO: Add example or reference

\subsection{Haskell}

Haskell is a lazy, functional, general-purpose programming language.
\cite{haskell_report2010}
It first appeared in 1990 \cite{HistoryOfHaskell2007}  and have since
released the major standards Haskell 98 and Haskell 2010
\cite{haskell_report2010}. Figure \ref{fig:simple_program} shows a simple program in Haskell.

\begin{figure}
  \begin{minted}{haskell}
main = print (fibonacci 10)

fibonacci :: Int -> Int
fibonacci 1 = 0
fibonacci 2 = 1
fibonacci x = fibonacci (x - 1) + fibonacci (x - 2)
  \end{minted}
  \caption{A simple Haskell program}
  \label{fig:simple_program}
\end{figure}


Two functions are defined in this program, \texttt{main} and
\texttt{fibonacci}.  The explicit type signature for the function
\texttt{fibonacci} means that it takes an int and returns an int. If a type
signature is omitted, like for \texttt{main}, Haskell will infer it automatically.

\subsubsection{Error handling in Haskell}

In order for stack traces to be relevant for a programming language, programs
must have the notion of \emph{crashing}. Intuitively, crashing causes sudden
stops in execution, either by the operating system or by the language's own
exception handling. Program crashes can be disastrous, since they will also
terminate processes that are supposed to be long-running. Hence there are
language constructs to eliminate some causes of program crashes.
For instance,
in well-typed Haskell programs, there are no segfaults \cite{FindingTheNeedle2009}.
% TODO: (behövs? källa?)   or any other time where the operating system
% directly intervenes. (really? source?)

In Haskell, there's a notion of a function being \emph{total}. Meaning that a
function will terminate and not return any error. % TODO: (source to definition?)
Therefor, such a function can not crash. Unfortunately, as of the famous
halting problem it's not possible to decide if a function will terminate or not, so it's not possible to say if a function is total.
% TODO (source: some really old paper by Turing?)
The good news, though, is that whenever we
explicitly \emph{choose} to crash, we can systematically avoid it. This is not
only true in the language Haskell, Java implements this through the
\texttt{throws} keyword. %TODO citera?
Figure \ref{fig:total_java} shows a Java integer division function that's
total.

\begin{figure}
  \begin{subfigure}[t]{1\textwidth}
      \begin{minted}{java}
int integerDivision (int nom, int den) throws ArithmeticException {
  if (den == 0) {
    throws ArithmeticException("Division by zero");
  }
  else {
    return nom / den;
  }
}
       \end{minted}
    \caption{A total function in Java}
    \label{fig:total_java}
  \end{subfigure}
        \vskip2em
        \begin{subfigure}[t]{1\textwidth}
          \begin{minted}{haskell}
integerDivision :: Int -> Int -> Maybe Int
integerDivision n 0 = Nothing
integerdivision n d = Just (n `div` d)
          \end{minted}
          \caption{A total function in Haskell}
          \label{fig:total_haskell}
        \end{subfigure}
        \caption{Two total functions
        }\label{fig:total_functions}
\end{figure}


In Haskell, it's instead convention to simply not return a value when there's
no valid value to return. To do this in Haskell, the \emph{Maybe} wrapper is
used to allow functions to gracefully fail, as illustrated in figure
\ref{fig:total_haskell}.
% TODO, eventuellt jämföra med Pythons None

The two functions in figure \ref{fig:total_functions} will not crash when dividing
by zero, rather, they gracefully return a value of either the division or a
value representing failure. But there's a drawback, both these functions are
cumbersome to use. In Java the programmer needs to explicitly catch the
Exception % TODO (citera?)
combining the \texttt{try} and \texttt{catch} constructs.
\cite{oracle_java_doc_catch} In Haskell, an additional layer of pattern
matching is required. Due to this inconvenience, both languages allow for
carrying out integer division without forcing the caller to do any error
handling. Figure \ref{fig:partial_functions} shows two partial functions.
% TODO fotnot om att det i java kallas RuntimeException när det intebehöver
% fångas?
% Alternativt kan man ju faktiskt ge länken till kontreversin
% http://docs.oracle.com/javase/tutorial/essential/exceptions/runtime.html
% om unchecked exceptions. För att visa att det jag pratar om är en riktig
% grej.

\begin{figure}
  \begin{subfigure}[t]{1\textwidth}
      \begin{minted}{java}
int integerDivisionUnsafe (int nom, int den) {
  return nom / den;
}
      \end{minted}
    \caption{A partial function in Java}
    \label{fig:partial_java}
  \end{subfigure}
        \vskip2em
        \begin{subfigure}[t]{1\textwidth}
          \begin{minted}{haskell}
integerDivisionUnsafe :: Int -> Int -> Int
integerDivisionUnsafe n 0 = error "Division by zero"
integerDivisionUnsafe n d = n `div` d
          \end{minted}
          \caption{A partial function in Haskell}
          \label{fig:partial_haskell}
        \end{subfigure}
        \caption{Two partial functions
        }\label{fig:partial_functions}
\end{figure}

For the first time we now see the \texttt{error} function in Haskell (figure \ref{fig:partial_haskell}).  It's a
special in-built function that terminates execution and outputs the provided
message. While it's not entirely accurate, we could think of \texttt{error}
being the only gateway to crashing a Haskell program. That means that all the
typical dangerous operations like integer division by zero or indexing outside
an array would just invoke the \texttt{error} function. We define ``crashing''
to be whenenver \texttt{error} is called.

% (footnote: It's
% rather the evaluation of an expression tree having \texttt{error} as it's top
% expression). % TODO: Behövs inte kanske.

% TODO: (ska man ha en formell Definition 1.1 blah blah här?)

\subsection{Glasgow Haskell Compiler}

The Glasgow Haskell Compiler (GHC) is a Haskell2010 compatible compiler.
\cite{ghc_website} With it you can compile Haskell source code to an executable
binary. Here's an invocation of the compiler on the program sample from figure \ref{fig:simple}.

\begin{minted}{bash}
  $ ghc --make Fibonacci.hs
  ...
  $ ./a.out
  34
\end{minted}

% GHC is the by far most used Haskell compiler (source). It has always (?)
% been in active (?) development since its first release in 199x (source) (Onödigt? Ta
% bort allt till och med hit?). Since then, many notable extensions have been
% added.
GHC as of today support many features in addition to the Haskell2010
standard, like parallelism, many optimizations, an llvm backend, profiling
support and more \cite{ghc_website}. This
work is adding low overhead stack traces to that list.

\subsubsection{The execution stack}

Most programmers have a decent picture of how programming languages implement
functions. They use the stack pointer register. Whenever a function is called, its
arguments are pushed on the stack by the caller and the caller jumps to the
functions code. When the function finally exits, it returns to where it was
called from. This was a short reminder of how the \emph{regular stack} works.
Most programming languages use this to implement the concept of functions.

Due to the nature of Haskell. It's not clear if the stack that worked so
elegantly for languages like C can be used to implement Haskell. How does it
work with partial applications? How does it work for lazyness? Instead one
might look at creating a completly new execution machine not based on a stack.
Peyton Jones created the Spineless Tagless G-Machine \cite{stg_1992}, an
execution machine which Haskell can be compiled down to. This machine \emph{is}
actually inspired by in the typical stack model. It's stack is called the
\emph{execution stack} and GHC implements this machine \cite{evalapplyjfp06}.
It's similar to stacks in typical programming languages. One property of this
stack is that a function written in the original source language (the Haskell
code) will have its own stack frame on the stack. Again, this is not a
necessity for a Haskell implementation. But it is the whole basis of my thesis.
If stack traces are going to be reconstructed by walking the stack, a stack
must exist! The take away here is that when we're talking about walking the
stack in this paper, we mean the execution stack as described in the STG, not
the traditional stack. Therefore my work only applies to GHC and not Haskell in
general.

% TODO: (bra bild på stacken)

% TODO: Att det är en länkad lista kanske ska komma senare I nästa del? Eller
% ska den komma här? Troligen här va? Typ som i Tereis paper

\subsection{From source to machine code}

Typically, compilers take source code and convert to machine code. To ease this
task, there usually are some intermediate representations (IR) in the source
code to machine code pipeline.  For GHC, the representations are illustrated in
figure TODO. It's outside the scope of this paper to look at how the IRs
look like. But the general strategy of succesively recasting source code to lower and
lower abstractions until it becomes machine code is very relevant to this work
and is not only happening in haskell compilers. (citera dragon book?)

There is one common complex problem that must be solved to enable debugging
tools: The programmer thinks of the program as it's source code and the
semantics of the language. However, the processor only runs machine code.
Unfortunately, there is by default no way to associate the machine code to the
source code that it originated from. This is a problem for all applications of
debugging, not limited to stack traces. (Cite: dwarf) As a consequence, any
compiler that wants to support debugging (footnote: without changing
performance) have to do the truly overwhelming task of threading along
information about the original source code that got compiled into each
intermediate step, this information must also be retained and tranformed
accordingly during all the optimization steps. In fact, this is impossible and
any implementation can only be a best effort implementation, figure TODO
motivates this.

Finally, the information  about the source-level functions that the compiler
have held tight throughout the IRs must get packaged into the binary. This
concern arises naturally in the final IR stage.  How does the compiler emit the
debug information? How is it stored in a way so it doesn't get in the way of the
actual code?  A debugging format answers these questions. One such debugging
format is called DWARF.

\subsection{DWARF}

In 1988, DWARF was created hoping to solve the quite general problem. DWARF is
a language agnostic debugging format that is still producing updated revisions.
DWARF 5 is planned to be released in 2014. (Cite: Dwarf) The DWARF data that is
stored in the binary can the be understood by a debugger like gdb. For example,
it could help gdb explain how some data should be displayed, for instance if a
particular byte is a 8-bit number or a character. % TODO fortsätt skriv!


(vad ska jag skriva här? Ännu oklart hur mycket DWARFmekande jag själv kommer
göra under arbetets gång ...)

% Skriva om varför debugging är svårt? Inspiration från DWARF-summarien
% Sedan vidare skriva om peters work, hans "causality" och sånt samt att allt
% egentligen är en best effort.

\chapter{Related work} \label{chp:related_work}

To programmers outside of the Haskell community, it could sound
surprising that a mature language like Haskell doesn't support stack
traces. This might raise the following questions:

\begin{itemize}
  \itemsep1pt\parskip0pt\parsep0pt
  \item
    Since stack traces are difficult, what other means of debugging are
    there?
  \item
    Are stack traces in Haskell really necessary?
  \item
    Is there at least any inefficient way to get stack traces?
  \item
    How close is the Haskell community in solving stack traces?
\end{itemize}

The overall structure of this chapter is that we answer the questions
by looking at related work. Most of the
related work is recent, usually less than 5 years from the time of
publishing of this work. The first question is answered in section
\ref{sec:debugging_haskell}. Section \ref{sec:avoiding_crashing}
shows that Haskell is a language producing robust programs, which
alleviates the need for stack traces.
The third question is answered in section
\ref{sec:overhead_full} which shows many working implementations of
stack traces, all of which have significant overhead. The last question
is answered in section \ref{sec:recent_work}.

\section{Debugging Haskell} \label{sec:debugging_haskell}

Examining stack traces falls in the category of debugging. Programmers
examine stack traces printed from a handled exception or a
crash. The amount of time the program runs before it crashes can be
anywhere between a few nanoseconds to many years. Stack traces are most
valuable for programs that crash unexpectedly after a long time of
stable execution, because it might be hard to reproduce the error in
order to diagnose it. Ideally, the stack trace aids the programmer in
writing a minimal reproducible test case that exercises the original
bug. Once programmers have an easy to reproduce bug, they look for
tools that help them better understand why the bug is happening. In this
subsection we'll look at existing tools for GHC that let programmers
step through the program's execution and even print variables. Neither of
which the stack trace implementation in this paper can provide.

\subsection{GCHi Debugger}

GHC comes with its own interactive \emph{read evaluate print loop} (REPL)
which has an built-in debugger. It's rich in features, supporting break
points, single-stepping, breaking on crashes, a "tracing mode" and even
variable inspection. The implementation works only with interpreted
code \cite{marlow2007lightweight, ghci_debugger}. So there will be significant overhead both
from the fact that the code is interpreted and that the debugger is
running.


\subsection{ghc-vis}

Debuggers are a view into the otherwise opaque executing program. The
GHCi debugger interface is text based, the programmer enters a command
to the debugger and it responds in text.  ghc-vis on the other hand is a
graphical debugger.  It allows users to visualize variables and interact
with them with the mouse pointer. For example, when clicking on an yet
unevaluated expression (remember, Haskell is a lazy language) it will
evaluate the expression.  Making it great for stepping through your
program without losing the big picture.  ghc-vis is hooking itself into
the program by running a thread inside of ghci. So it will only work
with ghci.  \cite{thesisFelsingBA}

\section{Avoiding Crashing} \label{sec:avoiding_crashing}

If a program never crashes,
it will not matter if our language prints stack traces or not. Never-crashing programs is a research area, sometimes called formal
verification. There are many approaches to formal verification. One
can statically analyze C-programs \cite{ckl2004},
use finite automata
or formal grammars \cite{dantam2013motion, rouhani2013software},
use type system tricks \cite{cheney2003first}
or use total functional programming \cite{Turner:jucs_10_7:total_functional_programming}.
In the end tough, none of the these methods are perfect, otherwise we
would not need stack traces.

There are also statical analysis tools for Haskell. HALO is a tool
where the programmers writes contracts about their own programs and
then lets HALO prove them \cite{vytiniotis2013halo}. HALO seems to
be inspired from \cite{xu2009static} which in turn is inspired by
\cite{xu2006extended}. Another tool is HipSpec which does automatic
proof finding instead of having the programmer spell out the properties
to validate \cite{claessen2013automating}. Yet another tool that's
mentioned on Haskellwiki is Catch where its described to "detect common
sources of runtime errors" \cite{haskellwiki_static_analysis_tools}.

\subsection{Catch}

Catch is a static analyzer for Haskell. It can detect if a pattern-matching is
sufficiently covering, even if the cases aren't collectively exhaustive. Figure
\ref{fig:catch_example} shows a function where the pattern match isn't exhaustive but
sufficiently so.
Catch can prove that such a pattern
matching is safe by doing flow analysis and ruling out impossible
patterns for the scrutiny (the expression that we \texttt{case} on).
This eliminates the need for the human programmer to manually check what can be
automatically proven. \cite{mitchell:catch_2008_9_25}

\begin{figure}
\begin{mdframed}
      \begin{minted}{haskell}
safeFunction = nonExhaustivePatterns False
  where
    nonExhaustivePatterns False = 42
      -- NOTE: No pattern for True
      \end{minted}
      \caption{A safe function even though the non-exhaustive matching. A
        totality checker like Catch can ensure that it's safe.}
      \label{fig:catch_example}
\end{mdframed}
\end{figure}

\section{Inefficient stack traces} \label{sec:overhead_full}

There are already many successful stack trace implementations in
Haskell. Unfortunately, they all have a significant overhead.
In this section we will look at previous work about stack traces for
Haskell. There are two common sources of overhead in existing implementations:

\begin{itemize}
\itemsep1pt\parskip0pt\parsep0pt
\item
  By building an explicit call stack (Subsection \ref{sec:explicit_call_stack})
\item
  By depending on expensive runtime settings (Subsection \ref{sec:stack_traces_with_profiling})
\end{itemize}

\subsection{Explicit call stack} \label{sec:explicit_call_stack}

Stack traces can be achieved by doing some methodological source level
transformations. Figure \ref{fig:transformation} shows a program transformed
into one producing stack traces on calls to \texttt{error}. This transformation is essentially:

\begin{itemize}
\itemsep1pt\parskip0pt\parsep0pt
\item
  Changing all top level functions to take one additional string
  argument. Except for the program entry-point \texttt{main}.
\item
  Transform all equations to define the new call stack \texttt{stack'} and
  pass it as the first argument to all calls of top level functions.
\item
  Transform all calls to \texttt{error} to also print out the call stack.
\end{itemize}

This transformation is similar to \cite{source_transformation} and a
complete source-to-source implementation called \emph{hat} exists already
\cite{hat_website}. But explicit call stack implementations don't need
to work on a source level.

\begin{figure}
\begin{mdframed}
        \begin{subfigure}[t]{0.4\textwidth}
            \begin{minted}{haskell}
main = print (f 100)

f :: Int -> Int
f x = g (5*x)

g :: Int -> Int
g 7 = error "Bang"
g x = 100 * x
            \end{minted}
            \caption{Original program}
        \end{subfigure}
        ~ %add desired spacing between images, e. g. ~, \quad, \qquad etc.
          %(or a blank line to force the subfigure onto a new line)
        \begin{subfigure}[t]{0.6\textwidth}
          \begin{minted}{haskell}
main = print (f stack' 100)
  where
    stack' = "main \n"

f :: String -> Int -> Int
f stack x = g stack' (5*x)
  where
    stack' = "f (case 1)\n" ++ stack

g :: String -> Int -> Int
g stack 7 = error ("Bang" ++ stack')
  where
    stack' = "g (case 1)\n" ++ stack
g stack x = 100 * x
  where
    stack' = "g (case 2)\n" ++ stack
          \end{minted}
          \caption{Transformed program}
        \end{subfigure}
        \caption{An example of how a Haskell program can be transformed to one
          that will print stack traces on errors. The syntax `\texttt{str1 ++
            str2}' is string concatenation.
        }\label{fig:transformation}
\end{mdframed}
\end{figure}

\subsubsection{StackTrace}

Allwood et al implemented a Intermediate Representation (IR) transformation
pass called \emph{StackTrace}. It's operating on the GHC Core IR. Since
Core is like a small subset of Haskell, its implementation will do
something similar to what figure \ref{fig:transformation} illustrates.

Among its complications are the
handling of higher order functions, linking with code that doesn't have stack
traces and an efficient non-naive implementation of the
passed along stack \cite{FindingTheNeedle2009}.
Functional programming in particular relies on efficient tail call
optimizations, which
requires the passed around call stack to efficiently handle this.

% TODO, Detta (ContMarks) passar inte in alltså, iom allt annat har med just
% Haskell att göra plus att det ska vara om overhead-full stack traces.
% detta borde ju bli overhead-full men svårt att bara påstå det.
% \subsubsection{Continuation Marks}

% The other dimensionality of complications is the implementation
% complexity.  There are other debugging features besides stack traces a
% compiler would like to support, and the more features the bigger blowup
% in the complexity of the compiler itself.  Clements disseratation shows
% a framework called \emph{continuation marks} which allows for a generic
% way to create addons to the language \cite{clements_dissertation2005}.
% In his disseratation, he uses continuation marks as a common ground of
% implementation for stack traces, code stepping in debugging and aspect
% oriented programming.  Since GHC does not have continuation marks
% implemented, it's not a possible starting point for implementing stack
% traces.

\subsection{Stack traces with profiling} \label{sec:stack_traces_with_profiling}

A mature and stable implementation of stack traces for
Haskell is present in GHC since GHC 7.4.1 which was released in February
2012. No paper has been produced from this effort. But a talk were
given at Haskell Implementors Workshop in September 2012 \cite{HIW2012Programme}.
The implementation is only working in conjunction with the profiling
mode of GHC. In Profiling mode the execution of programs can expect to
be twice as slow as their plain counterparts. The cost centre stack traces
have its own set of problems and is only an approximation of what Haskell
really is executing.

\section{Recent work} \label{sec:recent_work}

Around the time when this thesis started, Peter Wortmann, a
PhD candidate at University of Leeds showed a proof of concept
stack trace in Haskell that was based on the execution stack
\cite{stack_traces_ticket}. Peter had been working on non intrusive
profiling for GHC. To accomplish this, he had developed a theory of
causality of computations in Haskell and his work extended even to
optimized code \cite{DBLP:conf/haskell/WortmannD13}. To do profiling
he needed to map instruction pointers to the corresponding source
code. He added source code annotations that propagated
through the pipeline of IRs and optimizations and finally emitted DWARF
debugging data. Figure \ref{fig:core_and_ticks} shows how a code
annotation has been placed in the Core IR. With his patches, GHC now
emits DWARF, making a stack trace implementation to be low hanging
fruit, enabling the quite sizeable problem of stack traces to be worked
on during the limited scope of a master's thesis.

\begin{figure}
\begin{mdframed}
        \begin{subfigure}[t]{0.4\textwidth}
            \begin{minted}{text}
$wfibonacci =
  \ (ww_spO :: Int#) ->
    case ww_spO of ds_Xph {
      __DEFAULT -> ...
            \end{minted}
            \caption{Without ticks, the default option}
        \end{subfigure}
        ~ %add desired spacing between images, e. g. ~, \quad, \qquad etc.
          %(or a blank line to force the subfigure onto a new line)
        \begin{subfigure}[t]{0.6\textwidth}
          \begin{minted}{text}
$wfibonacci =
  \ (ww_spO :: Int#) ->
    src<0:Fibonnaci.hs:(4,1)-(6,51)>
    case ww_spO of ds_Xph {
      __DEFAULT -> ...
          \end{minted}
          \caption{With ticks by passing the flags \texttt{-g -dppr-ticks}}
        \end{subfigure}
        \caption{The same excerpt of the Core generated from the program
in figure \ref{fig:simple_program}. Only the version on the right have
debug data included }\label{fig:core_and_ticks}
\end{mdframed}
\end{figure}

But the original stack traces produced from Peter's simple demo
isn't satisfactory. The output from running the program in figure
\ref{fig:sample_program} could look like \ref{fig:trace_goal1}. This
work is partially motivated by trying to make the stack look more like
\ref{fig:trace_goal2}.

\begin{figure}
\begin{mdframed}
  \begin{minted}{haskell}
main :: IO ()
main = do print 1
          a
          print 2

a, b, c :: IO ()
a = do print 10
       b
       print 20

b = do print 100
       c
       print 200

c = do print 1000
       print (crashSelf 2)
       print 2000

crashSelf :: Int -> Int
crashSelf 0 = 1 `div` 0
crashSelf x = crashSelf (x - 1)
  \end{minted}
  \caption{A sample Haskell program that will crash when run}
  \label{fig:sample_program}
\end{mdframed}
\end{figure}




\begin{figure}
\begin{mdframed}
  \begin{subfigure}[t]{0.5\textwidth}
    {\small
    \begin{minted}{text}
 0: stg_bh_upd_frame_ret
 1: stg_bh_upd_frame_ret
 2: stg_bh_upd_frame_ret
 3: showSignedInt
 4: stg_upd_frame_ret
 5: writeBlocks
 6: stg_ap_v_ret
 7: bindIO
 8: bindIO
 9: bindIO
10: bindIO
11: stg_catch_frame_ret
    \end{minted}
  }%
    \caption{A stack trace, clearly based on the execution stack. Since the
      execution stack is bound to GHC's specific implementation of Haskell,
      it'll be difficult for programmers to interpret.}
    \label{fig:trace_goal1}
  \end{subfigure}
        ~ %add desired spacing between images, e. g. ~, \quad, \qquad etc.
          %(or a blank line to force the subfigure onto a new line)
        \begin{subfigure}[t]{0.5\textwidth}
    {\small
          \begin{minted}{text}
 0: crashSelf
 1: crashSelf
 2: print
 3: c
 4: b
 5: a
 6: main
          \end{minted}
  }%
          \caption{An ideal, fictive, stack trace, without
          implementation details of the execution stack. It's rather a
          semantic stack.}
          \label{fig:trace_goal2}
        \end{subfigure}
        \caption{Two stack traces
        }\label{fig:traces}
\end{mdframed}
\end{figure}

 % TODO, consider making chapter

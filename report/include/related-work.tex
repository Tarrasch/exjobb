\chapter{Related work} \label{chp:related_work}

To programmers outside of the Haskell community, it could sound
surprising that a mature language like Haskell doesn't support stack
traces. This might raise the following questions:

\begin{itemize}
  \itemsep1pt\parskip0pt\parsep0pt
  \item
    Since stack traces are difficult, what other means of debugging are
    there?
  \item
    Are stack traces in Haskell really necessary?
  \item
    Is there at least any inefficient way to get stack traces?
  \item
    How close is the Haskell community in solving stack traces?
\end{itemize}

The overall structure of this chapter is that we answer the questions
by looking at related work. Most of the
related work is recent, usually less than 5 years from the time of
publishing of this work. The first question is answered in section
\ref{sec:debugging_haskell}. Section \ref{sec:avoiding_crashing}
shows that Haskell is a language producing robust programs, which
alleviates the need for stack traces.
The third question is answered in section
\ref{sec:overhead_full} which shows many working implementations of
stack traces, all of which have significant overhead. The last question
is answered in section \ref{sec:recent_work}.

\section{Debugging Haskell} \label{sec:debugging_haskell}

Examining stack traces falls in the category of debugging. Programmers
examine stack traces printed from a handled exception or a
crash. The amount of time the program runs before it crashes can be
anywhere between a few nanoseconds to many years. Stack traces are most
valuable for programs that crash unexpectedly after a long time of
stable execution, because it might be hard to reproduce the error in
order to diagnose it. Ideally, the stack trace aids the programmer in
writing a minimal reproducible test case that exercises the original
bug. Once programmers have an easy to reproduce bug, they look for
tools that help them better understand why the bug is happening. In this
subsection we'll look at existing tools for GHC that let programmers
step through the program's execution and even print variables. Neither of
which the stack trace implementation in this paper can provide.

\subsection{GHCi Debugger}

GHC comes with its own interactive \emph{read evaluate print loop} (REPL)
which has a built-in debugger. It's rich in features, supporting break
points, single-stepping, breaking on crashes, a "tracing mode" and even
variable inspection. The implementation works only with interpreted
code \cite{marlow2007lightweight, ghci_debugger}. So there will be significant overhead both
from the fact that the code is interpreted and that the debugger is
running.


\subsection{ghc-vis}

Debuggers are a view into the otherwise opaque executing program. The
GHCi debugger interface is text based, the programmer enters a command
to the debugger and it responds in text.  ghc-vis on the other hand is a
graphical debugger.  It allows users to visualize variables and interact
with them with the mouse pointer. For example, when clicking on an yet
unevaluated expression (remember, Haskell is a lazy language) it will
evaluate the expression.  Making it great for stepping through your
program without losing the big picture.  ghc-vis is hooking itself into
the program by running a thread inside of GHCi. So it will only work
with GHCi \cite{thesisFelsingBA}.

\section{Avoiding Crashing} \label{sec:avoiding_crashing}

If a program never crashes,
it will not matter if our language prints stack traces or not. Never-crashing programs is a research area, sometimes called formal
verification. There are many approaches to formal verification. One
can statically analyze C-programs \cite{ckl2004},
use finite automata
or formal grammars \cite{dantam2013motion, rouhani2013software},
use type system tricks \cite{cheney2003first}
or use total functional programming \cite{Turner:jucs_10_7:total_functional_programming}.
In the end though, none of these methods are perfect, otherwise we
would not need stack traces.

There are also statical analysis tools for Haskell. HALO is a tool
where the programmers write contracts about their own programs and
then let HALO prove them \cite{vytiniotis2013halo}. HALO seems to
be inspired from \cite{xu2009static} which in turn is inspired by
\cite{xu2006extended}. Another tool is HipSpec which does automatic
proof finding instead of having the programmer spell out the properties
to validate \cite{claessen2013automating}. Yet another tool that's
mentioned on Haskellwiki is Catch where its described to "detect common
sources of runtime errors" \cite{haskellwiki_static_analysis_tools}.

\subsection{Catch}

Catch is a static analyzer for Haskell. It can detect if a pattern-matching is
sufficiently covering, even if the cases aren't collectively exhaustive. Figure
\ref{fig:catch_example} shows a function where the pattern match isn't exhaustive but
sufficiently so.
Catch can prove that such a pattern
matching is safe by doing flow analysis and ruling out impossible
patterns for the scrutiny (the expression that we \texttt{case} on).
This eliminates the need for the human programmer to manually check what can be
automatically proven \cite{mitchell:catch_2008_9_25}.

\begin{figure}
\begin{mdframed}
      \begin{minted}{haskell}
safeFunction = nonExhaustivePatterns False
  where
    nonExhaustivePatterns False = 42
      -- NOTE: No pattern for True
      \end{minted}
      \caption{A safe function even though the non-exhaustive matching. A
        totality checker like Catch can ensure that it's safe.}
      \label{fig:catch_example}
\end{mdframed}
\end{figure}

\section{Inefficient stack traces} \label{sec:overhead_full}

There are already many successful stack trace implementations in
Haskell. Unfortunately, they all have a significant overhead.
In this section we will look at previous work about stack traces for
Haskell. There are two common sources of overhead in existing implementations:

\begin{itemize}
\itemsep1pt\parskip0pt\parsep0pt
\item
  By building an explicit call stack (Subsection \ref{sec:explicit_call_stack})
\item
  By depending on expensive runtime settings (Subsection \ref{sec:stack_traces_with_profiling})
\end{itemize}

\subsection{Explicit call stack} \label{sec:explicit_call_stack}

Stack traces can be achieved by doing some methodological source level
transformations. Figure \ref{fig:transformation} shows a program transformed
into one producing stack traces on calls to \texttt{error}. This transformation is essentially:

\begin{itemize}
\itemsep1pt\parskip0pt\parsep0pt
\item
  Changing all top level functions to take one additional string
  argument. Except for the program entry-point \texttt{main}.
\item
  Transform all equations to define the new call stack \texttt{stack'} and
  pass it as the first argument to all calls of top level functions.
\item
  Transform all calls to \texttt{error} to also print out the call stack.
\end{itemize}

This transformation is similar to \cite{source_transformation} and a
complete source-to-source implementation called \emph{hat} exists already
\cite{hat_website}. But explicit call stack implementations don't need
to work on a source level.

\begin{figure}
\begin{mdframed}
        \begin{subfigure}[t]{0.4\textwidth}
            \begin{minted}{haskell}
main = print (f 100)

f :: Int -> Int
f x = g (5*x)

g :: Int -> Int
g 7 = error "Bang"
g x = 100 * x
            \end{minted}
            \caption{Original program}
        \end{subfigure}
        ~ %add desired spacing between images, e. g. ~, \quad, \qquad etc.
          %(or a blank line to force the subfigure onto a new line)
        \begin{subfigure}[t]{0.6\textwidth}
          \begin{minted}{haskell}
main = print (f stack' 100)
  where
    stack' = "main \n"

f :: String -> Int -> Int
f stack x = g stack' (5*x)
  where
    stack' = "f (case 1)\n" ++ stack

g :: String -> Int -> Int
g stack 7 = error ("Bang" ++ stack')
  where
    stack' = "g (case 1)\n" ++ stack
g stack x = 100 * x
  where
    stack' = "g (case 2)\n" ++ stack
          \end{minted}
          \caption{Transformed program}
        \end{subfigure}
        \caption{An example of how a Haskell program can be transformed to one
          that will print stack traces on errors. The syntax `\texttt{str1 ++
            str2}' is string concatenation.
        }\label{fig:transformation}
\end{mdframed}
\end{figure}

\subsubsection{StackTrace}

Allwood et al implemented a Intermediate Representation (IR) transformation
pass called \emph{StackTrace}. It's operating on the GHC Core IR. Since
Core is like a small subset of Haskell, its implementation will do
something similar to what figure \ref{fig:transformation} illustrates.

Among its complications are the
handling of higher order functions, linking with code that doesn't have stack
traces and an efficient non-naive implementation of the
passed along stack \cite{FindingTheNeedle2009}.
Functional programming in particular relies on efficient tail call
optimizations, which
requires the passed around call stack to efficiently handle this.

\subsection{Stack traces with profiling} \label{sec:stack_traces_with_profiling}

A mature and stable implementation of stack traces for
Haskell is present in GHC since GHC 7.4.1 which was released in February
2012. No paper has been produced from this effort. But a talk were
given at Haskell Implementors Workshop in September 2012 \cite{HIW2012Programme}.
The implementation is only working in conjunction with the profiling
mode of GHC. In Profiling mode the execution of programs can expect to
be twice as slow as their plain counterparts. The cost centre stack traces
have its own set of problems and is only an approximation of what Haskell
really is executing.

\section{Recent work} \label{sec:recent_work}

Around the time when this thesis started, Peter Wortmann, a
PhD candidate at University of Leeds showed a proof of concept
stack trace in Haskell that was based on the execution stack
\cite{stack_traces_ticket}. Peter had been working on non intrusive
profiling for GHC. To accomplish this, he had developed a theory of
causality of computations in Haskell and his work extended even to
optimized code \cite{DBLP:conf/haskell/WortmannD13}. To do profiling
he needed to map instruction pointers to the corresponding source
code. He added source code annotations that propagated
through the pipeline of IRs and optimizations and finally emitted DWARF
debugging data. Figure \ref{fig:core_and_ticks} shows how a code
annotation has been placed in the Core IR. With his patches, GHC now
emits DWARF, making a stack trace implementation to be low hanging
fruit, enabling the quite sizeable problem of stack traces to be worked
on during the limited scope of a master's thesis.

\begin{figure}
\begin{mdframed}
        \begin{subfigure}[t]{0.4\textwidth}
            \begin{minted}{text}
$wfibonacci =
  \ (ww_spO :: Int#) ->
    case ww_spO of ds_Xph {
      __DEFAULT -> ...
            \end{minted}
            \caption{Without ticks, the default option}
        \end{subfigure}
        ~ %add desired spacing between images, e. g. ~, \quad, \qquad etc.
          %(or a blank line to force the subfigure onto a new line)
        \begin{subfigure}[t]{0.6\textwidth}
          \begin{minted}{text}
$wfibonacci =
  \ (ww_spO :: Int#) ->
    src<0:Fibonnaci.hs:(4,1)-(6,51)>
    case ww_spO of ds_Xph {
      __DEFAULT -> ...
          \end{minted}
          \caption{With ticks by passing the flags \texttt{-g -dppr-ticks}}
        \end{subfigure}
        \caption{The same excerpt of the Core generated from the program
in figure \ref{fig:simple_program}. Only the version on the right have
debug data included }\label{fig:core_and_ticks}
\end{mdframed}
\end{figure}

But the original stack traces produced from Peter's simple demo
are not satisfactory. The stack trace from running the program in
figure \ref{fig:sample_program} looks like \ref{fig:trace_goal1}.
The outputted stack trace doesn't contain a single function from the
program in figure \ref{fig:sample_program}. The bottom of the stack is
\texttt{stg\_catch\_frame\_ret}, which is the default catch-handler that
is installed at the root of Haskell programs, before \texttt{main} is
run. The \texttt{writeBlocks} and \texttt{showSignedInt} give vague
hints that we're printing an \texttt{Int}-type, possibly to stdout
(they come from the usage of \texttt{print}). Parts of chapter
\ref{chp:reifying_the_stack} will look how to improve the stack. Ideally
the stack should look like in figure \ref{fig:trace_goal2}. Yet, we
have not looked deeply into the internals of GHC and its execution
stack, we will do this right now throughout the whole of chapter
\ref{chp:the_execution_stack}.

\begin{figure}
\begin{mdframed}
  \begin{minted}{haskell}
main :: IO ()
main = do print 1
          a
          print 2

a, b, c :: IO ()
a = do print 10
       b
       print 20

b = do print 100
       c
       print 200

c = do print 1000
       print (crashSelf 2)
       print 2000

crashSelf :: Int -> Int
crashSelf 0 = 1 `div` 0
crashSelf x = crashSelf (x - 1)
  \end{minted}
  \caption{A sample Haskell program that will crash when run}
  \label{fig:sample_program}
\end{mdframed}
\end{figure}




\begin{figure}
\begin{mdframed}
  \begin{subfigure}[t]{0.5\textwidth}
    {\small
    \begin{minted}{text}
 0: stg_bh_upd_frame_ret
 1: stg_bh_upd_frame_ret
 2: stg_bh_upd_frame_ret
 3: showSignedInt
 4: stg_upd_frame_ret
 5: writeBlocks
 6: stg_ap_v_ret
 7: bindIO
 8: bindIO
 9: bindIO
10: bindIO
11: stg_catch_frame_ret
    \end{minted}
  }%
    \caption{A stack trace, clearly based on the execution stack. Since the
      execution stack is bound to GHC's specific implementation of Haskell,
      it'll be difficult for programmers to interpret.}
    \label{fig:trace_goal1}
  \end{subfigure}
        ~ %add desired spacing between images, e. g. ~, \quad, \qquad etc.
          %(or a blank line to force the subfigure onto a new line)
        \begin{subfigure}[t]{0.5\textwidth}
    {\small
          \begin{minted}{text}
 0: crashSelf
 1: crashSelf
 2: print
 3: c
 4: b
 5: a
 6: main
          \end{minted}
  }%
          \caption{An ideal, fictive, stack trace, without
          implementation details of the execution stack. It's rather a
          semantic stack.}
          \label{fig:trace_goal2}
        \end{subfigure}
        \caption{Two stack traces
        }\label{fig:traces}
\end{mdframed}
\end{figure}


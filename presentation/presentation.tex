\documentclass[11pt]{beamer} % mathserif for normal math fonts.
\usefonttheme[onlymath]{serif}
\usepackage[utf8]{inputenc}
\usepackage[swedish,english]{babel}
\usepackage{amsmath,mathtools}
\usepackage{calc}
\usepackage{minted}

\usepackage[T1]{fontenc}

% The Chalmers theme:
\usetheme[titleflower=true]{chalmers} % titleflower = true or false
\title{Stack Traces in Haskell}
% \subtitle{And something more} % optional % I dont need it /Arash
\author[Arash Rouhani]{Arash Rouhani} % [short author (optional)]{many authors}
\institute{Chalmers University of Technology}
\titlepageextra{Master thesis presentation} % Optional extra info, appears before date on title page
\footer{\insertshortauthor\ -- Thesis presentation} % optional, manually sets footer (default is short author)
%\footer{Something completely different} % but it can of course be anything.
\titlepagelogofile{figures/Avancez_black} % File name to the file you want to include.
%\titlepagelogo{\tikz{\draw(0,0) circle (1);}} % or draw anything for a logo


% https://3diagramsperpage.wordpress.com/2013/10/26/template-for-code-highlighting-with-minted-in-latex-beamer/
% http://alturl.com/689br
\newminted{haskell}{fontsize=\scriptsize,
		   linenos,
		   numbersep=8pt,
		   gobble=2,
		   frame=lines,
		   bgcolor=bg,
		   framesep=3mm
       }

\newminted{text}{fontsize=\scriptsize,
		   gobble=2,
		   bgcolor=bg
		   }


\begin{document}
% http://alturl.com/689br
\definecolor{bg}{rgb}{0.95,0.95,0.95}

% http://alturl.com/689br
\defverbatim[colored]\exampleCode{
\begin{haskellcode}
  hi = world

\end{haskellcode}
}


%\setlength{\footertextwidth}{0.5\paperwidth} % You might need to adjust this if the text doesn't fit well.

\section{Title page} % Sections are shown at the bottom left. There is also links in many pdf-readers
\begin{frame}[plain]
 \titlepage
\end{frame}

\section{Contents}
\begin{frame}
 \frametitle{Contents}
 % \framesubtitle{With subtitle}
\begin{itemize}
 \item Motivation
 \item Background
 \item Contribution
\end{itemize}
\end{frame}

\section{Motivation}

  \begin{frame}
   \frametitle{An old problem \dots}
  \begin{itemize}
   \item <1-> Try running this program:
     \motivationCode
   \item <2-> You get
     \outputNoTrace
   \item <3-> But you want
     \outputTrace
  \end{itemize}
  \end{frame}

  \begin{frame}
   \frametitle{\dots with new constraints}
  \begin{itemize}
   \item Should have very low overhead
   \item If you hesitate to use it in production, I've failed
   \item Not done for Haskell before
  \end{itemize}
  \end{frame}

\section{Background}

  \begin{frame}
   \frametitle{Background contents}
  \begin{itemize}
   \item Is stack traces harder for \emph{Haskell}?
   \item Will the implementation only work for \emph{GHC}?
  \end{itemize}
  \end{frame}


\subsection{Haskell}

  \begin{frame}
   \frametitle{Laziness}
  \begin{itemize}
   \item Consider the code
     \lazyCode
  \end{itemize}
  \end{frame}

  \begin{frame}
   \frametitle{Case expressions}
  \begin{itemize}
   \item <1-> Consider the code
     \caseCode
   \item <1-> So is pattern matching just like \texttt{switch-case} in C?
   \item <2-> NO!
   \item <3-> \texttt{myBool} can be a delayed computation, aka a \emph{thunk}
  \end{itemize}
  \end{frame}

\subsection{GHC}
  \begin{frame}
   \frametitle{History of GHC}
  \begin{itemize}
   \item Compiles Haskell to machine code since 1989
   \item The only one people care about
  \end{itemize}
  \end{frame}

  \begin{frame}
   \frametitle{Usage}
  \begin{itemize}
   \item Compile and run (just like any other compiler)
     \useGhcCode
  \end{itemize}
  \end{frame}

\subsubsection{Magic function}
  \begin{frame}
   \frametitle{The magical function}
  \begin{itemize}
   \item <1-> My work assumes there the existence of
     \getDebugInfoCode
   \item <2-> This is a recent contribution not yet merged in HEAD 
   \item <2-> Author is
     \href{http://www.personal.leeds.ac.uk/~scpmw/site.html}{Peter Wortmann},
     part of his PhD at Leeds
   \item <3-> \emph{In essence, 95\% of the job to implement stack traces was
       already done!}
  \end{itemize}
  \end{frame}

  \begin{frame}
   \frametitle{The compilation pipeline}
  \begin{itemize}
   \item <1-> Well GHC works like this:
     \includegraphics[width=4in]{build/fig/ghc}
   \item <2-> Or rather like this
     \includegraphics[width=4in]{build/fig/phases}
   \item <3-> We say that GHC has many \emph{Intermediete Representations}
  \end{itemize}
  \end{frame}

  \begin{frame}
   \frametitle{So there must be debug data!}
  \begin{itemize}
   \item <1-> Again:
     \includegraphics[width=4in]{build/fig/ghc}
   \item <2-> The intuition behind \texttt{getDebugInfo} is:
     \includegraphics[width=4in]{build/fig/ghc-reverse}
   \item <3-> For this, we \emph{must} retain debug data in the binary!
  \end{itemize}
  \end{frame}

  \begin{frame}
   \frametitle{Lets get to work!}
   \includegraphics[width=2.2in]{fig/recastings}
   \includegraphics[width=2.2in]{fig/recastings_ticks}
  % \begin{itemize}
  %  \item <1-> Recall:
  %    \includegraphics[width=3in]{build/fig/phases}
  %  \item <2> Lol:
  %    % TODO, paralell transformations
  % \end{itemize}
  \end{frame}

  \begin{frame}
   \frametitle{What happened?}
  % trim=l b r t
   \includegraphics[trim=0 0.5in 0 7.8in, clip=true, width=2.2in]{fig/recastings}
   \includegraphics[trim=0 0 0 8.5in, clip=true, width=2.2in]{fig/recastings_ticks}
  \begin{itemize}
   \item Did we just drop the debug data we worked so hard for?
  \end{itemize}
  \end{frame}

  \begin{frame}
   \frametitle{This is a solved problem, of course!}
  \begin{itemize}
   \item <1-> DWARF to the rescue!
    \dwarfCode
   \item <2-> DWARF lives \emph{side by side} in another section of the binary.
     Therefor it does not interfer.
  \end{itemize}
  \end{frame}

\subsubsection{The Stack}

  \begin{frame}
   \frametitle{Introduction to the Execution Stack}
  \begin{itemize}
   \item <1-> GHC \emph{chooses} to implement Haskell with a stack.
   \item <2-> It does not use the normal ``C-stack''
   \item <3-> GHC maintains its own stack, we call it the \emph{execution stack}.
  \end{itemize}
  \end{frame}

  \begin{frame}
   \frametitle{Similar but not same}
  \begin{itemize}
   \item <1-> Unlike C, we do not push something on the stack when entering a function!
   \item <2-> Unlike C, we have cheap green threads, one stack per thread!
  \end{itemize}
  \end{frame}

  \begin{frame}
   \frametitle{What is on it then?}
  \begin{itemize}
   \item <1-> Recall this code:
     \caseCode
   \item <2-> How is this implemented? Let's think for a while \dots
   \item <3-> Aha! We can push a \emph{continuation} on the stack and jump to the code of \texttt{myBool}!
   \item <4-> We call this a \emph{case continuation}.
  \end{itemize}
  \end{frame}

\section{The breakthrough in August 2013}

  \begin{frame}
   \frametitle{Peter's demonstration}
  \begin{itemize}
   \item <1-> In August 2013 Peter Wortmann showed a proof of concept stack
     trace based on his work.
   \item <2-> My master thesis is entirely based on Peter's work.
  \end{itemize}
  \end{frame}

  \begin{frame}
   \frametitle{The stack trace \dots}
  \begin{itemize}
   \item For this Haskell code:
     \HaskellSTCode
  \end{itemize}
  \end{frame}

  \begin{frame}
   \frametitle{\dots is \emph{terrible}!}
  \begin{itemize}
   \item <1-> We get:
     \OriginalTraceCode
   \item <2-> We want:
     \IdealTraceCode
  \end{itemize}
  \end{frame}

\section{Contribution}
\begin{frame}
 \frametitle{}
\end{frame}

\begin{frame}
\frametitle{Code example}
\exampleCode
\pause
Overlays work!
\end{frame}

\end{document}
